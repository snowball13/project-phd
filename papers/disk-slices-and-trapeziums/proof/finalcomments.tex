\documentclass[11pt, oneside]{article}   	% use "amsart" instead of "article" for AMSLaTeX format
\usepackage{geometry}                		% See geometry.pdf to learn the layout options. There are lots.
\geometry{letterpaper}                   		% ... or a4paper or a5paper or ... 
%\geometry{landscape}                		% Activate for rotated page geometry
\usepackage[parfill]{parskip}    		% Activate to begin paragraphs with an empty line rather than an indent
\usepackage{graphicx}				% Use pdf, png, jpg, or eps§ with pdflatex; use eps in DVI mode
								% TeX will automatically convert eps --> pdf in pdflatex		
\usepackage[margin=10pt,font=footnotesize,labelfont=bf]{caption}
\usepackage{subcaption}
\usepackage{float}
\usepackage{amssymb}
\usepackage{amsmath}
\usepackage{bm}
\usepackage{bbm}
\usepackage{mleftright}
\usepackage{todonotes}

\include{somacros}

\begin{document}

Queries:

1) Yes 

2) Yes

3) Reference Figure 1 after Definition 4.
	"... and on the triangle. An illustration of how the non-weighted differential operators increment the parameters (a,b,c) is seen in Figure 1"
	
4) OK

5) Should be ``2020, IMA. J. Numer. Anal., https://doi.org/10.1093/imanum/draa001''

6) Replace the problematic sentence with the following:
	"Recall that the OPs $\mathbb{H}^{(a,b,c)}$ are orthogonal with respect to the weight $W^{(a,b,c)}$ on $\Omega$, and define the matrix $\Lambda^{(a,b,c)} := \ip< \mathbb{H}^{(a,b,c)}, \: {\mathbb{H}^{(a,b,c)}}^\top >_{W^{(a,b,c)}}$"
	
	
Typos:

1) In Lemma 1, it should say 
$$
\alpha^{(a,b,c,d)}_{n,k,1} := \beta_{n-k-1}^{(a, b, c+d+2k+1)}, \qquad \alpha^{(a,b,c,d)}_{n,k,2} := \alpha_{n-k}^{(a, b, c+d+2k+1)}
$$

2) At the end of Section 3, we are missing a couple of minus signs. We should have:
\begin{align*}
	\eta_j &= - \frac{1}{\beta^{(a,b,c,d)}_{n,j-1,9}} \big( \beta^{(a,b,c,d)}_{n,n+j+1,7} \: \eta_{j+2} + \beta^{(a,b,c,d)}_{n,n+j,8} \: \eta_{j+1} \big) \qfor j = n-1,n-2,\dots,1, \\
	\eta_0 &= - \frac{1}{\alpha^{(a,b,c,d)}_{n+1,0,1}} \big( \beta^{(a,b,c,d)}_{n,1,7} \: \eta_{2} + \beta^{(a,b,c,d)}_{n,0,8} \: \eta_{1} \big).
\end{align*}

3) Near the end of the proof for Theorem 1, it should read:
	"... nonzero coefficients $c^x_{m,j}$ are when $m = n-3, n-2, n-1$ and $j = k-2, k$"
(and not $m = n-3,...,n$)

4) Near the end of Section 3, it should read:
	"... for the Laplacian $\Delta$ that will take us from the coefficients for expansion in the space $\mathbb{H}^{(0,0,0)}$ to coefficients in the space $\mathbb{H}^{(2,2,2)}$"
(and not $\mathbb{H}^{(2,2)}$)

5) Similarly, it should read:
	"... Laplacian as a map from coefficients in the space $\mathbb{W}^{(2,2,2)}$ to coefficients in the space $\mathbb{H}^{(0,0,0)}$. Note a function expanded in the $\mathbb{W}^{(2,2,2)}$ basis will satisfy ..."
(and not $\mathbb{W}^{(2,2)}$)


Additional changes for Figures:

1) Figure 3b will be a new figure, that increases the N value used in the original. We would also then need to update the text in Section 5.1 that reads "N = 200, that is, 20,301 unknowns" to instead be "N = 990, that is, 491,536 unknowns"

2) Figure 5b will be a new figure, that increases the N value used in the original. We would also then need to update the text in Section 5.1 that reads "N = 200, that is, 20,301 unknowns" to instead be "N = 500, that is, 125,751 unknowns" (Note: Depends how high I can go before I run out of memory with the factorization/time)

3) Figure 6b will be a new figure, that increases the N value used in the original. We would also then need to update the text in Section 5.1 that reads "N = 200, that is, 20,301 unknowns" to instead be "N = 500, that is, 125,751 unknowns" (Note: Depends how high I can go before I run out of memory with the factorization/time)


\end{document}

