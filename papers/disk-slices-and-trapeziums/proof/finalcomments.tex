\documentclass[11pt, oneside]{article}   	% use "amsart" instead of "article" for AMSLaTeX format
\usepackage{geometry}                		% See geometry.pdf to learn the layout options. There are lots.
\geometry{letterpaper}                   		% ... or a4paper or a5paper or ... 
%\geometry{landscape}                		% Activate for rotated page geometry
\usepackage[parfill]{parskip}    		% Activate to begin paragraphs with an empty line rather than an indent
\usepackage{graphicx}				% Use pdf, png, jpg, or eps§ with pdflatex; use eps in DVI mode
								% TeX will automatically convert eps --> pdf in pdflatex		
\usepackage[margin=10pt,font=footnotesize,labelfont=bf]{caption}
\usepackage{subcaption}
\usepackage{float}
\usepackage{amssymb}
\usepackage{amsmath}
\usepackage{bm}
\usepackage{bbm}
\usepackage{mleftright}
\usepackage{todonotes}


\def\addtab#1={#1\;&=}

\def\meeq#1{\def\ccr{\\\addtab}
%\tabskip=\@centering
 \begin{align*}
 \addtab#1
 \end{align*}
  }  
  
  \def\leqaddtab#1\leq{#1\;&\leq}
  \def\mleeq#1{\def\ccr{\\\addtab}
%\tabskip=\@centering
 \begin{align*}
 \leqaddtab#1
 \end{align*}
  }  


\def\vc#1{\mbox{\boldmath$#1$\unboldmath}}

\def\vcsmall#1{\mbox{\boldmath$\scriptstyle #1$\unboldmath}}

\def\vczero{{\mathbf 0}}


%\def\beginlist{\begin{itemize}}
%
%\def\endlist{\end{itemize}}


\def\pr(#1){\left({#1}\right)}
\def\br[#1]{\left[{#1}\right]}
\def\fbr[#1]{\!\left[{#1}\right]}
\def\set#1{\left\{{#1}\right\}}
\def\ip<#1>{\left\langle{#1}\right\rangle}
\def\iip<#1>{\left\langle\!\langle{#1}\right\rangle\!\rangle}

\def\norm#1{\left\| #1 \right\|}

\def\abs#1{\left|{#1}\right|}
\def\fpr(#1){\!\pr({#1})}

\def\Re{{\rm Re}\,}
\def\Im{{\rm Im}\,}

\def\floor#1{\left\lfloor#1\right\rfloor}
\def\ceil#1{\left\lceil#1\right\rceil}


\def\mapengine#1,#2.{\mapfunction{#1}\ifx\void#2\else\mapengine #2.\fi }

\def\map[#1]{\mapengine #1,\void.}

\def\mapenginesep_#1#2,#3.{\mapfunction{#2}\ifx\void#3\else#1\mapengine #3.\fi }

\def\mapsep_#1[#2]{\mapenginesep_{#1}#2,\void.}


\def\vcbr[#1]{\pr(#1)}


\def\bvect[#1,#2]{
{
\def\dots{\cdots}
\def\mapfunction##1{\ | \  ##1}
	\sopmatrix{
		 \,#1\map[#2]\,
	}
}
}

\def\vect[#1]{
{\def\dots{\ldots}
	\vcbr[{#1}]
}}

\def\vectt[#1]{
{\def\dots{\ldots}
	\vect[{#1}]^{\top}
}}

\def\Vectt[#1]{
{
\def\mapfunction##1{##1 \cr} 
\def\dots{\vdots}
	\begin{pmatrix}
		\map[#1]
	\end{pmatrix}
}}



\def\thetaB{\mbox{\boldmath$\theta$}}
\def\zetaB{\mbox{\boldmath$\zeta$}}


\def\newterm#1{{\it #1}\index{#1}}


\def\TT{{\mathbb T}}
\def\C{{\mathbb C}}
\def\R{{\mathbb R}}
\def\II{{\mathbb I}}
\def\F{{\mathcal F}}
\def\E{{\rm e}}
\def\I{{\rm i}}
\def\D{{\rm d}}
\def\dx{\D x}
\def\dy{\D y}
\def\CC{{\cal C}}
\def\DD{{\cal D}}
\def\U{{\mathbb U}}
\def\A{{\cal A}}
\def\K{{\cal K}}
\def\DTU{{\cal D}_{{\rm T} \rightarrow {\rm U}}}
\def\LL{{\cal L}}
\def\B{{\cal B}}
\def\T{{\cal T}}
\def\W{{\cal W}}


\def\tF_#1{{\tt F}_{#1}}
\def\Fm{\tF_m}
\def\Fab{\tF_{\alpha,\beta}}
\def\FC{\T}
\def\FCpmz{\FC^{\pm {\rm z}}}
\def\FCz{\FC^{\rm z}}

\def\tFC_#1{{\tt T}_{#1}}
\def\FCn{\tFC_n}

\def\rmz{{\rm z}}

\def\chapref#1{Chapter~\ref{Chapter:#1}}
\def\secref#1{Section~\ref{Section:#1}}
\def\exref#1{Exercise~\ref{Exercise:#1}}
\def\lmref#1{Lemma~\ref{Lemma:#1}}
\def\propref#1{Proposition~\ref{Proposition:#1}}
\def\warnref#1{Warning~\ref{Warning:#1}}
\def\thref#1{Theorem~\ref{Theorem:#1}}
\def\defref#1{Definition~\ref{Definition:#1}}
\def\probref#1{Problem~\ref{Problem:#1}}
\def\corref#1{Corollary~\ref{Corollary:#1}}
\def\appref#1{Appendix~\ref{Appendix:#1}}

\def\sgn{{\rm sgn}\,}
\def\Ai{{\rm Ai}\,}
\def\Bi{{\rm Bi}\,}
\def\wind{{\rm wind}\,}
\def\erf{{\rm erf}\,}
\def\erfc{{\rm erfc}\,}
\def\qqquad{\qquad\quad}
\def\qqqquad{\qquad\qquad}


\def\spand{\hbox{ and }}
\def\spodd{\hbox{ odd}}
\def\speven{\hbox{ even}}
\def\qand{\quad\hbox{and}\quad}
\def\qqand{\qquad\hbox{and}\qquad}
\def\qfor{\quad\hbox{for}\quad}
\def\qqfor{\qquad\hbox{for}\qquad}
\def\qas{\quad\hbox{as}\quad}
\def\qqas{\qquad\hbox{as}\qquad}
\def\qor{\quad\hbox{or}\quad}
\def\qqor{\qquad\hbox{or}\qquad}
\def\qqwhere{\qquad\hbox{where}\qquad}



%%% Words

\def\naive{na\"\i ve\xspace}
\def\Jmap{Joukowsky map\xspace}
\def\Mobius{M\"obius\xspace}
\def\Holder{H\"older\xspace}
\def\Mathematica{{\sc Mathematica}\xspace}
\def\apriori{apriori\xspace}
\def\WHf{Weiner--Hopf factorization\xspace}
\def\WHfs{Weiner--Hopf factorizations\xspace}

\def\Jup{J_\uparrow^{-1}}
\def\Jdown{J_\downarrow^{-1}}
\def\Jin{J_+^{-1}}
\def\Jout{J_-^{-1}}



\def\bD{\D\!\!\!^-}




\def\questionequals{= \!\!\!\!\!\!{\scriptstyle ? \atop }\,\,\,}

\def\elll#1{\ell^{\lambda,#1}}
\def\elllp{\ell^{\lambda,p}}
\def\elllRp{\ell^{(\lambda,R),p}}


\def\elllRpz_#1{\ell_{#1{\rm z}}^{(\lambda,R),p}}


\def\sopmatrix#1{\begin{pmatrix}#1\end{pmatrix}}

\def\socases#1{\begin{cases} #1 \end{cases}}


\def\Problem#1#2\par{\begin{problem}\label{pb:#1} #2\end{problem}}
\def\Theorem#1#2\par{\begin{theorem}\label{th:#1} #2\end{theorem}}
\def\Conjecture#1#2\par{\begin{conjecture}\label{conj:#1} #2\end{conjecture}}
\def\Proposition#1#2\par{\begin{proposition}\label{prop:#1} #2\end{proposition}}
\def\Definition#1#2\par{\begin{definition}\label{def:#1} #2\end{definition}}
\def\Corollary#1#2\par{\begin{corollary}\label{cr:#1} #2\end{corollary}}
\def\Lemma#1#2\par{\begin{lemma}\label{lm:#1} #2\end{lemma}}
\def\Example#1#2\par{\begin{example}\label{ex:#1} #2\end{example}}
\def\Remark #1\par{\begin{remark*}#1\end{remark*}}


\def\Proof{\begin{proof}}
\def\mqed{\end{proof}}


\def\Figuretwow[#1,#2]#3#4\par{
\begin{figure}[tb]
\begin{center}{
\includegraphics[width=#3]{Figures/#1}\includegraphics[width=#3]{Figures/#2}}
\end{center}
\caption{#4}\label{fig:#1} 
\end{figure}
}

\begin{document}

Queries:

1) Yes 

2) Yes

3) Reference Figure 1 after Definition 4.
	"... and on the triangle. An illustration of how the non-weighted differential operators increment the parameters (a,b,c) is seen in Figure 1"
	
4) OK

5) Should be ``2020, IMA. J. Numer. Anal., https://doi.org/10.1093/imanum/draa001''

6) Replace the problematic sentence with the following:
	"Recall that the OPs $\mathbb{H}^{(a,b,c)}$ are orthogonal with respect to the weight $W^{(a,b,c)}$ on $\Omega$, and define the matrix $\Lambda^{(a,b,c)} := \ip< \mathbb{H}^{(a,b,c)}, \: {\mathbb{H}^{(a,b,c)}}^\top >_{W^{(a,b,c)}}$"
	
	
Typos:

1) In Lemma 1, it should say 
$$
\alpha^{(a,b,c,d)}_{n,k,1} := \beta_{n-k-1}^{(a, b, c+d+2k+1)}, \qquad \alpha^{(a,b,c,d)}_{n,k,2} := \alpha_{n-k}^{(a, b, c+d+2k+1)}
$$

2) At the end of Section 3, we are missing a couple of minus signs. We should have:
\begin{align*}
	\eta_j &= - \frac{1}{\beta^{(a,b,c,d)}_{n,j-1,9}} \big( \beta^{(a,b,c,d)}_{n,n+j+1,7} \: \eta_{j+2} + \beta^{(a,b,c,d)}_{n,n+j,8} \: \eta_{j+1} \big) \qfor j = n-1,n-2,\dots,1, \\
	\eta_0 &= - \frac{1}{\alpha^{(a,b,c,d)}_{n+1,0,1}} \big( \beta^{(a,b,c,d)}_{n,1,7} \: \eta_{2} + \beta^{(a,b,c,d)}_{n,0,8} \: \eta_{1} \big).
\end{align*}

3) Near the end of the proof for Theorem 1, it should read:
	"... nonzero coefficients $c^x_{m,j}$ are when $m = n-3, n-2, n-1$ and $j = k-2, k$"
(and not $m = n-3,...,n$)

4) Near the end of Section 3, it should read:
	"... for the Laplacian $\Delta$ that will take us from the coefficients for expansion in the space $\mathbb{H}^{(0,0,0)}$ to coefficients in the space $\mathbb{H}^{(2,2,2)}$"
(and not $\mathbb{H}^{(2,2)}$)

5) Similarly, it should read:
	"... Laplacian as a map from coefficients in the space $\mathbb{W}^{(2,2,2)}$ to coefficients in the space $\mathbb{H}^{(0,0,0)}$. Note a function expanded in the $\mathbb{W}^{(2,2,2)}$ basis will satisfy ..."
(and not $\mathbb{W}^{(2,2)}$)


Additional changes for Figures:

1) Figure 3b will be a new figure, that increases the N value used in the original. We would also then need to update the text in Section 5.1 that reads "N = 200, that is, 20,301 unknowns" to instead be "N = 990, that is, 491,536 unknowns"

2) Figure 5b will be a new figure, that increases the N value used in the original. We would also then need to update the text in Section 5.1 that reads "N = 200, that is, 20,301 unknowns" to instead be "N = 500, that is, 125,751 unknowns" (Note: Depends how high I can go before I run out of memory with the factorization/time)

3) Figure 6b will be a new figure, that increases the N value used in the original. We would also then need to update the text in Section 5.1 that reads "N = 200, that is, 20,301 unknowns" to instead be "N = 500, that is, 125,751 unknowns" (Note: Depends how high I can go before I run out of memory with the factorization/time)


\end{document}

