\documentclass[11pt, oneside]{article}   	% use "amsart" instead of "article" for AMSLaTeX format
\usepackage{geometry}                		% See geometry.pdf to learn the layout options. There are lots.
\geometry{letterpaper}                   		% ... or a4paper or a5paper or ... 
%\geometry{landscape}                		% Activate for rotated page geometry
\usepackage[parfill]{parskip}    		% Activate to begin paragraphs with an empty line rather than an indent
\usepackage{graphicx}				% Use pdf, png, jpg, or eps§ with pdflatex; use eps in DVI mode
								% TeX will automatically convert eps --> pdf in pdflatex		
\usepackage{caption}
\usepackage{subcaption}
\usepackage{float}
\usepackage{amssymb}
\usepackage{amsmath}
\usepackage{bm}
\usepackage{bbm}
\usepackage{mleftright}
\usepackage{todonotes}

\newcommand{\sotodo}{\todo[color=green]}
\newcommand{\sotodoinline}{\todo[color=green,inline=true]}
\newcommand{\bstodo}{\todo[color=pink]}
\newcommand{\bstodoinline}{\todo[color=pink,inline=true]}


%SetFonts

%SetFonts

\usepackage{natbib}
\usepackage{url}
%\bibliographystyle{elsarticle-harv}
\bibliographystyle{plain}

\newcommand{\half}{\frac{1}{2}}
\newcommand{\R}{\mathbb{R}}
\newcommand{\C}{\mathbb{C}}
\newcommand{\Z}{\mathbb{Z}}
\newcommand{\N}{\mathbb{N}}
\newcommand{\No}{\mathbb{N}_0}
\newcommand{\Ylm}{Y^m_l}
\newcommand{\Ylmfull}{Y^m_l(\theta,\varphi)}
\newcommand{\Plm}{P^m_l}
\newcommand{\costheta}{\cos\theta}
\newcommand{\sintheta}{\sin\theta}
\newcommand{\cosphi}{\cos\varphi}
\newcommand{\sinphi}{\sin\varphi}
\newcommand{\eimphi}{e^{im\varphi}}
\newcommand{\alphalm}{\alpha^m_l}
\newcommand{\clm}{c^m_l}
\newcommand{\ctilde}{\tilde{c}^m_l}
\newcommand{\ctildemod}{\tilde{c}^{|m|}_l}
\newcommand{\chat}{\hat{c}^m_l}
\newcommand{\chatmod}{\hat{c}^{|m|}_l}
\newcommand{\ddx}{\frac{\mathrm{d}}{\mathrm{d}x}}
\newcommand{\dmdxm}{\frac{\mathrm{d}^m}{\mathrm{d}x^m}}

\newcommand{\Atilde}{\tilde{A}_{l,m}}
\newcommand{\Btilde}{\tilde{B}_{l,m}}
\newcommand{\Dtilde}{\tilde{D}_{l,m}}
\newcommand{\Etilde}{\tilde{E}_{l,m}}
\newcommand{\Ftilde}{\tilde{F}_{l,m}}
\newcommand{\Gtilde}{\tilde{G}_{l,m}}
\newcommand{\Alm}{A_{l,m}}
\newcommand{\Blm}{B_{l,m}}
\newcommand{\Dlm}{D_{l,m}}
\newcommand{\Elm}{E_{l,m}}
\newcommand{\Flm}{F_{l,m}}
\newcommand{\Glm}{G_{l,m}}

\newcommand{\xione}{\xi^{(1)}_{n, \lambda}}
\newcommand{\xitwo}{\xi^{(2)}_{n, \lambda}}
\newcommand{\xithree}{\xi^{(3)}_{n, \lambda}}
\newcommand{\xifour}{\xi^{(4)}_{n, \lambda}}

\newcommand{\bigP}{\mathbb{P}}
\newcommand{\Pl}{\mathbb{P}_l}
\newcommand{\gradP}{T\mathbb{P}}
\newcommand{\gradPl}{T\mathbb{P}_l}
\newcommand{\gradY}{\nabla Y}
\newcommand{\gradYlm}{\nabla Y^m_l}
\newcommand{\gradpY}{\nabla^\perp Y}
\newcommand{\gradpYlm}{\nabla^\perp Y^m_l}

\newcommand{\Dlt}{D^T_l}

\newcommand{\curlyy}{\bm{\mathcal{Y}}}
\newcommand{\blone}{\beta_{l, 1}}
\newcommand{\blzero}{\beta_{l, 0}}
\newcommand{\blmone}{\beta_{l, -1}}
\newcommand{\chivec}{\bm{\chi}_{1,m_s}}
\newcommand{\cgcoeff}{\mathcal{C}}

\newcommand{\alm}{a_{l,m}}
\newcommand{\blm}{b_{l,m}}
\newcommand{\dlm}{d_{l,m}}
\newcommand{\elm}{e_{l,m}}
\newcommand{\flm}{f_{l,m}}
\newcommand{\glm}{g_{l,m}}
\newcommand{\hlm}{h_{l,m}}
\newcommand{\jlm}{j_{l,m}}
\newcommand{\klm}{k_{l,m}}
\newcommand{\almperp}{a_{l,m}^\perp}
\newcommand{\blmperp}{b_{l,m}^\perp}
\newcommand{\dlmperp}{d_{l,m}^\perp}
\newcommand{\elmperp}{e_{l,m}^\perp}
\newcommand{\flmperp}{f_{l,m}^\perp}
\newcommand{\glmperp}{g_{l,m}^\perp}
\newcommand{\hlmperp}{h_{l,m}^\perp}
\newcommand{\jlmperp}{j_{l,m}^\perp}
\newcommand{\klmperp}{k_{l,m}^\perp}

\newcommand{\unitvec}{\hat{\bm{k}}}

\newcommand{\Pnk}{P_{n,k}}
\newcommand{\Pnkab}{P_{n,k}^{(a,b)}}
\newcommand{\Wab}{{W^{(a,b)}}}
\newcommand{\Pmjab}{P_{m,j}^{(a,b)}}
\newcommand{\alphaab}{\alpha^{(a,b)}}
\newcommand{\betaab}{\beta^{(a,b)}}
\newcommand{\bigPab}{\bigP^{(a,b)}}
\newcommand{\Dnt}{D^T_n}
\newcommand{\Wii}{W^{(1,1)}}
\newcommand{\Pii}{P^{(1,1)}}
\newcommand{\Poo}{P^{(0,0)}}
\newcommand{\Poi}{P^{(0,1)}}
\newcommand{\Pio}{P^{(1,0)}}
\newcommand{\dx}{\frac{\partial}{\partial x}}
\newcommand{\dy}{\frac{\partial}{\partial y}}

%\usepackage{media9}
%\graphicspath{ {../sphere/plots} }



\title{Early Stage Assessment}
\author{Ben Snowball (Principle Supervisor: Sheehan Olver)}
\date{}							% Activate to display a given date or no date


\begin{document}

\maketitle




\section{Introduction}

The goal of my research is to eventually investigate sparse spectral methods on triangulations of the sphere as an alternative to the spherical harmonics approach currently in use at the European Centre for Medium-range Weather Forecasts (ECMWF) in their weather and climate model \cite{wedi2013fast}. The aim would be to preserve the predictive skill of the current approach whilst avoiding the parallel scalability bottleneck from the global spectral transform, which is expected to inhibit the future performance of the ECMWF model.

This document is to provide a brief outline of my work to date, with a closer look at our most recent area of research (solving PDEs on the half disk using sparse spectral methods). The work outlined in this report serves as initial steps to pave the way to solving PDEs on spherical triangles. All code used thus far and from hence forth will mainly be written in the Julia programming language.

\section{Professional Skills Development Requirement}
I have met the initial part of the professional skills development requirement which is to attend a minimum of 2 professional development workshops plus the online plagiarism awareness course by the Early Stage Assessment (ESA).  This is by virtue of the fact that I attended the Communication Skills Training in the MPE CDT on 03/11/2017.  This course has been given accreditation with the Graduate School programme and is worth 2 credits towards the requirement.


\section{Orthogonal polynomials on the sphere (spherical harmonics)}

The first step was to develop a small code library so that we can use spherical harmonics as orthogonal polynomials in three variables $x, y, z$  to evaluate functions on the sphere. 

We can the define the spherical harmonics, orthogonal on the unit sphere as \cite{DLMFSphericalandSpheroidalHarmonics, DLMFFerrers, DLMFAssociatedLegendre, efthimiou2014spherical, dai2013approximation}:
\begin{align}
\Ylmfull &:= \clm \eimphi \Plm (\costheta) \\
&= \alphalm (1 - (\costheta)^2)^\frac{|m|}{2} \eimphi P^{(|m|,|m|)}_{l-|m|}(\costheta), \quad \text{where } 0 \le |m| \le l, \, l \in \No.
\end{align} 
Note that we can then express \(\Ylm\) in terms of \(x,y,z\) instead of \(\theta, \varphi\) by noting that \(\costheta = z\) and that \( \eimphi\) can be expressed in terms of \(x,y,z\) for any \(m\in\mathbb{Z}\). Indeed, they are polynomials in $x,y,z$ which we denote $\Ylm(x,y,z)$. They span all polynomials modulo the ideal generated by $x^2 + y^2 + z^2 - 1$. While this is equivalent to standard techniques in the litrature concerning spherical harmonics, we approach this with our goal in mind of developing sparse spectral methods on spherical triangles. Thus, we try to use language and formulations that will translate as we move through our steps towards that goal.

\citep{dunkl2014orthogonal} (p75-85) provided a framework for gaining a recurrence relation for each \(\bigP_{l+1}\) in terms of the previous two sub-vectors \(\Pl\) and \(\bigP_{l-1}\), utilising the sub matrices that make up the block-tridiagnonal Jacobi operator matrices for multiplication by $x, y$ and $z$. Clenshaw's algorithm then gives us a way of evaluating a function given its expansion in the spherical harmonic basis (up to a finite order). This provides an optimal complexity method for evaluating spherical harmonic expansions; an alternative to using the explicit expressions that may have benefits, though further investigation is required.

\subsection{Deriving expressions for the multiplication by \(x, y, z\) of the spherical harmonics}

We start by expressing \(x\,\Ylm(x,y,z)\), \(y\,\Ylm(x,y,z)\) , and \(z\,\Ylm(x,y,z)\)  in terms of \(Y^{m'}_{l'}(x,y,z)\) for any point \((x,y,z)\) on the unit circle.

For \(l \in \No\), \(m \in \Z\) s.t. \(0 \le |m| \le l\):
\begin{align}
x\,\Ylm(x,y,z) &= \Alm Y^{m+1}_{l+1}(x,y,z) +  \Blm Y^{m+1}_{l-1}(x,y,z) \nonumber \\
		     & \quad \quad \quad + \Dlm Y^{m-1}_{l+1}(x,y,z) + \Elm Y^{m-1}_{l-1}(x,y,z), \\
y\,\Ylm(x,y,z) &= - i \, \big[\Alm Y^{m+1}_{l+1}(x,y,z) +  \Blm Y^{m+1}_{l-1}(x,y,z) \big] \nonumber \\
		     &\quad \quad \quad + i \, \big[ \Dlm Y^{m-1}_{l+1}(x,y,z) + \Elm Y^{m-1}_{l-1}(x,y,z) \big], \\
z\,\Ylm(x,y,z) &= \Flm Y^{m}_{l+1}(x,y,z) + \Glm Y^{m}_{l-1}(x,y,z) ,
\end{align}
where \(\Alm, \Blm, \Dlm, \Elm, \Flm, \Glm \in \R\) have explicit derivable and calculable formulae.

\subsection{Jacobi matrices}

Define, for \(l \in \No\), \(\Pl\) as the column vector of the degree \(l\) spherical harmonic polynomials, and \(\bigP\) as the stacked block vector of the \(\Pl\)'s; that is
\begin{align}
\Pl := \begin{bmatrix}
		Y^{-l}_l \\
		\vdots \\
		Y^l_l
	\end{bmatrix} \in \C^{2l+1}, 
\quad \quad 
\bigP := \begin{bmatrix}
		\bigP_0 \\
		\hline
		\bigP_1 \\
		\hline
		\bigP_2 \\
		\vdots \\
	\end{bmatrix}.
\end{align}

Define the (Jacobi) matrices \(J^x, J^y, J^z\) by 
\begin{align}
J^x \bigP = x \bigP, \quad J^y \bigP = y \bigP, \quad J^z \bigP = z \bigP.
\end{align}
Then, using equations (3--5), we have that the Jacobi matrices each have the following block-tridiagonal form, allowing for easy application computationally:
\begin{align}
J^{x/y/z} &= \begin{bmatrix}
		B^{x/y/z}_0 & A^{x/y/z}_0 & & & & \\
		C^{x/y/z}_1 & B^{x/y/z}_1 & A^{x/y/z}_1 & & & \\
		& C^{x/y/z}_2 & B^{x/y/z}_2 & A^{x/y/z}_2  & & & \\
		& & C^{x/y/z}_3 & \ddots & \ddots & \\
		& & & \ddots & \ddots & \ddots \\
	\end{bmatrix}
\end{align}

\subsection{Three-term recurrence relation for \(\bigP\)}

Combining each system in (7) we can write the tridiagonal-block system
\begin{align}
\renewcommand\arraystretch{1.3}
\mleft[
\begin{array}{cccc}
		1  & & & \\
		B_0-G_0(x,y,z) & A_0 & & \\
		C_1 & B_1-G_1(x,y,z) & \quad A_1 \quad & \\
		& C_2 & \ddots & \ddots \\
		& & \ddots &
\end{array}
\mright]
\bigP
=
\begin{bmatrix}
	\alpha^0_0 \\ 0 \\ 0 \\ \vdots \\ \\
\end{bmatrix}.
\end{align}
where for each \(l \in \No\):
\begin{align}
A_l &:= \begin{bmatrix}
		A^x_l \\
		A^y_l \\
		A^z_l
	    \end{bmatrix} \in \R^{3(2l+1)\times(2l+3)}, \quad
C_l := \begin{bmatrix}
		C^x_l \\
		C^y_l \\
		C^z_l
	    \end{bmatrix} \in \R^{3(2l+1)\times(2l-1)} \quad (l \ne 0), \\
B_l &:= \begin{bmatrix}
		B^x_l \\
		B^y_l \\
		B^z_l
	    \end{bmatrix} \in \R^{3(2l+1)\times(2l+1)}, \quad
G_l(x,y,z) := \begin{bmatrix}
		xI_{2l+1} \\
		yI_{2l+1} \\
		zI_{2l+1}
	    \end{bmatrix} \in \R^{3(2l+1)\times(2l+1)}.
\end{align}

For each \(l \in \No\) let \(\Dlt\) be any matrix that is a left inverse of \(A_l\), i.e. such that \(\Dlt A_l = I_{2l+3}\). Multiplying our system by the preconditioner matrix that is given by the block diagonal matrix of the \(\Dlt\)'s, we obtain a lower triangular system \citep{dunkl2014orthogonal} (p78), which can be expanded to obtain the recurrence:
\begin{align}
\begin{cases}
\bigP_{-1} := 0 \\
\bigP_{0} := \alpha^0_0 \\
\bigP_{l+1} = -\Dlt (B_l-G_l) \Pl - \Dlt C_l  \,\bigP_{l-1}, \quad l \in \N.
\end{cases}
\end{align}

\subsection{Evaluation of a scalar function on the sphere}
We can use a multivariate analogue of the Clenshaw algorithm to evaluate a function at a given point \((x,y,z)\) on the unit sphere provided we know the coefficients of the function when expanded in the spherical harmonic basis, i.e. suppose \(f(x,y,z)\) is a function and we know the set \(\{\bold{f}_l\}\) s.t.
\begin{align}
f(x,y,z) \approx \sum^N_{l=0} \bold{f}_l^T \, \Pl (x,y,z), \quad \text{where } \Pl (x,y,z), \bold{f}_l \in \R^{2l+1} \text{ for each } l \in \{0,\dots,N\}.
\end{align}

The Clenshaw algorithm is then as follows:
\begin{align*}
\quad &\text{1) } \text{Set } \bm{\gamma}_{N+2} = \bold{0}, \: \bm{\gamma}_{N+2} = \bold{0}. \\
\quad &\text{2) } \text{For } n = N:-1:1 \\
\quad & \quad \quad \quad \text{set } \bm{\gamma}_{n}^T = \bold{f}_n^T - \bm{\gamma}_{n+1}^T D^T_n (B_n - G_n(x,y,z)) -  \bm{\gamma}_{n+2}^T D^T_{n+1}C_{n+1} \\
\quad &\text{3) } \text{Output: } f(x,y,z) \approx \bigP_0(x,y,z) f_0 + \bm{\gamma}_{1}^T \bigP_1(x,y,z) - \bigP_0(x,y,z) \bm{\gamma}_{2}^T D^T_1 C_1.
\end{align*}
where $\bigP_0 \equiv \alpha^0_0$.

We can use the Clenshaw algorithm to output a matrix operator for a function, given we know the coefficients of its expansion in the SH basis. We simply make the adjustment that
\begin{align*}
\quad &\text{1) } \text{Set } \bm{\gamma}_{N+2} = \bold{0}, \: \bm{\gamma}_{N+2} = \bold{0}. \\
\quad &\text{2) } \text{For } n = N:-1:1 \\
\quad & \quad \quad \quad \text{set } \bm{\gamma}_{n}^T = (\bold{f}_n^T)I - \bm{\gamma}_{n+1}^T D^T_n (B_n - G_n(J^x,J^y,J^z)) -  \bm{\gamma}_{n+2}^T D^T_{n+1}C_{n+1} \\
\quad &\text{3) } \text{Output: } f(J^x,J^y,J^z) \approx F := \alpha^0_0 f_0 + \bm{\gamma}_{1}^T \bigP_1(J^x,J^y,J^z) - \alpha^0_0 \bm{\gamma}_{2}^T D^T_1 C_1.
\end{align*}
where $\bigP_1(J^x,J^y,J^z)$ is a vector of matrices, $G_n(J^x,J^y,J^z)$ is a matrix of matrices, and so each $\bm{\gamma}_n$ is a vector of matrices.

\subsection{Tangent space}

By noting that the tangent space is then spanned by $\{\gradYlm, \gradpYlm\}$, and that these gradients and perpendicular gradients are orthogonal polynomials themselves, we can use the same ideas to create a framework for functions in the tangent space as expansions in this OP basis. Note that the perpendicular gradient is related to the regular surface gradient by
\begin{align}
\gradpYlm(x,y,z) = \unitvec \times \gradYlm(x,y,z),
\end{align}
where \(\unitvec\) is the unit vector normal to the surface of the sphere at the point \((x,y,z)\), i.e. as we are looking at the unit sphere, \(\unitvec\) is simply given by
\begin{align}
\unitvec = \begin{bmatrix} x \\ y \\ z \end{bmatrix}.
\end{align}
For any scalar spherical harmonic (SSH) $\Ylm$, the triad of $\gradYlm, \: \unitvec \times \gradYlm, \: \unitvec \Ylm$ can be referred to as vector spherical harmonics \cite{BarreraEtAl}.

Since the spherical harmonics are a basis for the surface of the sphere, and the tangent space of the sphere is spanned by the gradient  and perpendicular gradient of a scalar function, we have that the gradients and perpendicular gradients of the spherical harmonics are a basis for the tangent space, namely \(\gradYlm\), \(\gradpYlm\). 

Importantly,  \(\gradYlm\), \(\gradpYlm\) are in fact vector-valued orthogonal polynomials, that span all vector-valued polynomials modulo the vanishing ideal $\{ \bm{p} \in (\Pi_3)^3  : \unitvec \cdot \bm{p} = 0 \}$. A simple calculation shows that such orthogonal polynomials must still have block-tridiagonal Jacobi operators, as multiplication by $x, y$, or $z$ remains inside the ideal. In the following, we determine the coefficients of these Jacobi operators, as well as other important operators needed.

\subsubsection{Spin-1 tensor spherical harmonics}

The gradient and perpendicular gradient of a spherical harmonic \(\Ylm(x,y,z)\) can be expressed in terms of spin-1 tensor spherical harmonics, which in turn can each be expressed as a vector-weighted sum of spherical harmonics.

For any \(l \in \No\), \(m \in \Z\) s.t. \(0 \le |m| \le l\) we have that
\begin{align}
\nabla \Ylm &= \blmone \, \curlyy^{l-1,1}_{l,m} + \blone \curlyy^{l+1,1}_{l,m}, \\
\nabla^\perp \Ylm &= \blzero \, \curlyy^{l,1}_{l,m},
\end{align}
where
\begin{align}
\blmone &:= (l+1) \, \Big(\frac{l}{2l+1}\Big)^\half, \quad
\blzero := i \, \big(l(l+1)\big)^\half, \quad
\blone := l \, \Big(\frac{l+1}{2l+1}\Big)^\half.
\end{align}

We can then gain similar (vector analogue) derivations for recurrences, Jacobi matrices and the Clenshaw algorithm implementation for the tangent space vector-valued orthogonal polynomial basis.

The aim is to, given a PDE say, work with the coefficients of the functions involved (coefficients of their expansion in the OP bases), reframe the PDE in terms of the OP coefficients utilising a suitable timestepping method. For derivative, gradient, divergence etc. operators, we can formulate matrix operators to act on the coefficients, thus making our system to simulate a simple matrix vector system.

\subsection{Example: linear shallow water equations}
Let $\mathbf{u}(x,y,z)$ be the tangential velocity of the flow and $h(x,y,z)$ be the height of the water from some constant reference height $\mathcal{H}$. The linear SWEs are
\begin{align}
\begin{cases}
\frac{\partial \mathbf{u}}{\partial t} + f \unitvec \times \mathbf{u} - \nabla h = \mathbf{0} \\
\frac{\partial h}{\partial t} + \mathcal{H} \nabla \cdot \mathbf{u} = 0
\end{cases}
\end{align}
where $f = 2 \Omega \cos(\theta) =  2 \Omega z$ is the Coriolis parameter (note that the operator on the coefficients vector for multiplication by this parameter would be $F = 2 \Omega (J^z_\nabla)^T$).

Using backward Euler:
\begin{align}
\mathbf{u}_{n+1} &= \mathbf{u}_{n} + \Delta t \: (\nabla h_{n+1} - f \unitvec \times \mathbf{u}_{n+1}) \\
h_{n+1} &= h_n - \Delta t \: \mathcal{H} \nabla \cdot \mathbf{u}_{n+1}
\end{align}

Then using the operators on the coefficient vectors, the timestepping method is: 
\begin{align}
\mathbf{u}^c_{n+1} &= \mathbf{u}^c_{n} + \Delta t \: (G \mathbf{h}^c_{n+1} - F K \mathbf{u}^c_{n+1}) \\
\mathbf{h}^c_{n+1} &= \mathbf{h}^c_n - \Delta t \: \mathcal{H} D \mathbf{u}^c_{n+1}
\end{align}
$\iff$
\begin{align}
\mathbf{u}^c_{n+1} &= (I + \Delta t F K + \Delta t^2 \mathcal{H} G D)^{-1} \: (\mathbf{u}^c_{n} + \Delta t \: G \mathbf{h}^c_n) \\
\mathbf{h}^c_{n+1} &= \mathbf{h}^c_n - \Delta t \: \mathcal{H} D \mathbf{u}^c_{n+1}
\end{align}
where $F, K, G, D$ are all operators corresponding to multiplying by the Coriolis parameter, outward normal vector cross product, gradient and divergence operations respectively. It is important to note that these can be easily constructed due to the relationships between the vector spherical harmonics and scalar spherical harmonics we have outlined, and have a sparse Block-Banded structure.

These ideas will also be used to gain similar relationships for the circle arc, half disk, and spherical triangle orthogonal polynomials.


\section{Orthogonal polynomials on the circle arc}

As a stepping stone to working on spherical triangles, we will try to construct orthogonal polynomials on a circle arc, given by $x \in [h, 1]$, $y = (1-x^2)^\frac{1}{2}$. We will use the Lanczos algorithm as implemented in the ApproxFun package \cite{ApproxFun} to obtain 1D OPs $\{T^h_n\}$ (orthogonal wrt $w(x) = (1-x^2)^{-\frac{1}{2}}$) and $\{U^h_n\}$ (orthogonal wrt $w(x) = (1-x^2)^{\frac{1}{2}}$) on the interval $[h,1]$, and create our 2D OPs according to \cite{huybrechs2010fourier}:
\begin{align}
\bigP_0(x,y) := T^h_0(x) \equiv T_0,
\quad \quad 
\bigP_n(x,y) := \begin{bmatrix}
		T^h_n(x) \\
		y \: U^h_{n-1}(x)
	\end{bmatrix} \in \R^2 \quad \forall n \in \N,
\quad \quad 
\bigP := \begin{bmatrix}
		\bigP_0 \\
		\hline
		\bigP_1 \\
		\hline
		\bigP_2 \\
		\vdots \\
	\end{bmatrix}.
\end{align}


\section{Orthogonal polynomials on the half disk (2D)}

By using a similar process to \cite{dunkl2014orthogonal} (p55-56) we can construct 2D orthogonal polynomials on the half disk from 1D orthogonal polynomials on the intervals \([0,1]\) and \([-1,1]\). We denote the half disk by \(\Omega\) where
\begin{align}
\Omega := \{(x,y) \in \R^2 \quad | \quad 0 < x < 1, \: -(1-x^2)^{\half} < y < (1-x^2)^{\half}\}.
\end{align}

Let \(w_1 : (\alpha,\beta) \: \to \R\), \(w_2 : (\gamma,\delta) \: \to \R\) be weight functions with \(\alpha,\beta,\gamma,\delta \in \R\), and \(\rho \: : \: (\alpha,\beta) \: \to (0,\infty)\) be such that either:

1) \(\rho\) is a degree 1 polynomial, or 

2) \(\rho\) is the square root of a non-negative degree \(\le\) 2 polynomial, \(-\gamma = \delta > 0\), and \(w_2\) is an even function.

Then, \(\forall n,k = 0,1,2,\dots\) let \(\{p_{n,k}\}\) be polynomials orthogonal with respect to the weight \(\rho(x)^{2k+1} w_1(x)\), and \(\{q_{n}\}\) be polynomials orthogonal with respect to the weight \(w_2(x)\).

We define \(\Pnk(x,y) := p_{n-k,k}(x) \: \rho(x)^k \: q_k(\frac{y}{\rho(x)})\) for \( 0 \le k \le n, \: n = 0,1,2,\dots\) and note that the functions \(\{\Pnk\}\) are in fact polynomials in either cases above. Furthermore, they are orthogonal with respect to the weight \(W(x,y) := w_1(x) \: w_2(\frac{y}{\rho(x)}) \) on \(R := \{(x,y) \in \R^2 \quad | \quad \alpha < x < \beta, \: \gamma \rho(x) < y < \delta \rho(x)\}\).

Let \(a, b \in \R\). For the half disk, if we set:

\begin{align}
\alpha &:= 0 \\
\beta &:= 1 \\
\gamma &:= -1 \\
\delta &:= -c = 1 \\
\rho(x) &:= (1-x^2)^{\half} \\
w_1(x) &:= x^a \: (1-x^2)^b \\
w_2(x) &:= (1-x^2)^b = (1-x)^b \: (1+x)^b,
\end{align}

we can obtain the 1D orthogonal polynomials \(\{H_n^{(a, b)}\}\) orthogonal with respect to the weight \(x^a \: (1-x^2)^b\), and \(\{P_n^{(b, b)}\}\) orthogonal with respect to the weight \((1-x^2)^b\) (so \(\{P_n^{(b, b)}\}\) are in fact Jacobi polynomials), leading to the two parameter 2D orthogonal polynomials \(\{\Pnk\}\) given by, for \(0 \le k \le n, \: n = 0,1,2,\dots,\)
\begin{align}
 \Pnk(x,y) := H_{n-k}^{(a, b+k+\half)}(x) \: \rho(x)^k \: P_k^{(b,b)}(\frac{y}{\rho(x)}), \quad (x,y) \in \Omega, 
\end{align}
orthogonal on \(\Omega\) with respect to the weight 
\begin{align}
W^{(a,b)}(x,y) := x^a \: (1-x^2-y^2)^b, \quad (x,y) \in \Omega.
\end{align}

This provides us with a family of orthogonal polynomial spaces on the half disk.


\subsection{Obtaining the coefficients for expansion of a function}

Fix \(a,b \in \R\). Then for any function \(f : \Omega \to \R\) we can approximate (or exactly express) \(f\) by
\begin{align}
f(x,y) \approx \sum_{n=0}^N \bigP_n^{(a,b)}(x,y)^T \: \mathbf{f}_n
\end{align}
where
\begin{align}
\bigP(x,y) := \begin{bmatrix}
		\bigP_0(x,y) \\
		\hline
		\bigP_1(x,y) \\
		\hline
		\bigP_2(x,y) \\
		\hline
		\vdots \\
		\hline
		\bigP_N(x,y)
	\end{bmatrix},
\quad
\bigP_n(x,y) := \begin{bmatrix}
		P_{n,0}(x,y) \\
		\vdots \\
		P_{n,n}(x,y)
	\end{bmatrix} \in \R^{n+1} \quad \forall n = 0,1,2,\dots,N,
\end{align}
and where
\begin{align}
\mathbf{f} &:= \begin{bmatrix}
		\mathbf{f}_0 \\
		\hline
		\mathbf{f}_1 \\
		\hline
		\mathbf{f}_2\\
		\hline
		\vdots \\
		\hline
		\mathbf{f}_N
	\end{bmatrix},
\quad
\mathbf{f}_n := \begin{bmatrix}
		f_{n,0} \\
		\vdots \\
		f_{n,n}
	\end{bmatrix} \in \R^{n+1} \quad \forall n = 0,1,2,\dots,N, \\
f_{n,k} &:= \frac{\langle f, \: \Pnk^{(a,b)} \rangle_{W^{(a,b)}}}{|| \Pnk^{(a,b)} ||_{W^{(a,b)}}}
\end{align}




We can develop a quadrature rule on \(\Omega\) in order to evaluate (40) and obtain the coefficients vector \(\mathbf{f}\) as follows. 

We will use the substitution that
\begin{align}
x &= s, \quad y = (1-s^2)^\half t.
\end{align}
First, note that
\begin{align}
W^{(a,b)}(x,y) &= x^a \: (1-x^2-y^2)^b, \quad \text{for } (x,y) \in \Omega, \\
		      &= s^a \: (1-s^2)^{b} \: (1-t^2)^b =: V^{(a,b)}(s,t), \quad \text{for } (s,t) \in [0,1] \times [-1,1].
\end{align}
Using the Golub-Welsch algorithm, we can obtain nodes and weights for 1D quadrature rules on \([0,1]\) with weight \(s^a \: (1-s^2)^{b+\half}\), and on \([-1,1]\) with weight \((1-t^2)^b\).

Let \(f(x,y)\) be polynomial that is even in \(y\) of degree $d_1$ in $x$ and degree $d_2$ in $y$. Then there exists a polynomial \(\tilde{f}(x,y)\) in \(x,y\) such that \(g(s,t) := f(x, y) = f(s, (1-s^2)^\half t) = \tilde{f}(s, (1-s^2) t^2)\) (i.e. that $d_2$ is an even integer). Now, as \(\tilde{f}\) is a polynomial, so must \(g\) be a polynomial of degree $d_1+\frac{d_2}{2}$ in $s$ and degree $d_2$ in $t$. Thus a quadrature rule of the following will be exact for the polynomial $f(x,y)$ (even in \(y\) of degree $\le d_1$ in $x$ and degree $\le d_2$ in $y$):
\begin{align}
& \quad \int_0^1 \int_{-\rho(x)}^{\rho(x)} W^{(a,b)}(x,y) \: f(x,y) \: dy \: dx \\
&= \int_0^1 \int_{-1}^1 V^{(a,b)}(s,t) \: g(s,t) \: (1-s^2)^\half \: dt \: ds \\
&= \int_0^1 s^a \: (1-s^2)^{b+\half} \: \Big( \sum_k^{\half(d_2 + 1)} w_k^{(t)} \: g(s, t_k) \Big) \: ds \\
&= \sum_j^{\half(d_1 + \frac{d_2}{2} + 1)} \sum_k^{\half(d_2 + 1)} w_j^{(s)} w_k^{(t)} \: g(s_j, t_k) \\
&= \sum_j^{\half(d_1 + \frac{d_2}{2} + 1)} \sum_k^{\half(d_2 + 1)} w_j^{(s)} w_k^{(t)} \: f(x_j, y_{j,k}),
\end{align}
where \(x_j := s_j,\) \(y_{j,k} := \rho(s_j) \: t_k\) are the quadrature points.

We further note that for any polynomial \(f(x,y)\) that is odd in \(y\),
\begin{align}
\int_0^1 \int_{-\rho(x)}^{\rho(x)} W^{(a,b)}(x,y) \: f(x,y) \: dy \: dx = 0.
\end{align}

Relabelling the weights and nodes as 
\begin{align}
x_{i+(k-1)N} &:= s_k, \quad i,k = 1,\dots,N, \\
y_{l+(i-1)N} &:= (1-s_l^2)^\half \: t_l, \quad i,l = 1,\dots,N, \\
w_{l+(k-1)N} &:= w_k^{(s)} w_l^{(t)}, \quad k,l = 1,\dots,N.
\end{align}

Then our quadrature rule for any polynomial \(f(x,y)\) on \(\Omega\) is:
\begin{align}
\int \int_\Omega W^{(a,b)}(x,y) \: f(x,y) \: dy \: dx \approx \half \: \sum_{j=1}^{N^2} w_j \: \big[ f(x_j, y_j) + f(x_j, -y_j) \big],
\end{align}
and is exact for any polynomial $f(x,y)$ where the degree in $x$ is $\le N$ and the degree in $y$ is $\le 2N-1$. Thus,
\begin{align}
< f, \: \Pnk^{(a,b)} >_{W^{(a,b)}} &= \half \: \sum_{j=1}^{N^2} w_j \: \big[ \big(f(x_j, y_j) \: \Pnk^{(a,b)}(x_j, y_j)\big) +\big(f(x_j, -y_j) \: \Pnk^{(a,b)}(x_j, -y_j)\big) \big], \\
|| \Pnk^{(a,b)} ||_{W^{(a,b)}}^2 &= \half \: \sum_{j=1}^{N^2} w_j \: \big[ \Pnk^{(a,b)}(x_j, y_j)^2 + \Pnk^{(a,b)}(x_j, -y_j)^2 \big],
\end{align}
for suitably large $N$.


\subsection{Jacobi matrices}

We obtain the 1D OPs via the Lanczos algorithm, which also gives us the coefficients of the relations
\begin{align}
x H_n^{(a,b)}(x) &= \beta_n^{(a,b)} H_{n+1}^{(a,b)}(x) + \alpha_n^{(a,b)} H_n^{(a,b)}(x) + \beta_{n-1}^{(a,b)} H_{n-1}^{(a,b)}(x) \\
y P_n^{(b,b)}(y) &= \delta_n^{(b)} P_{n+1}^{(b,b)}(y) + \gamma_n^{(b)} P_n^{(b,b)}(y) + \delta_{n-1}^{(b)} P_{n-1}^{(b,b)}(y),
\end{align}
where we note that \(\gamma_n^{(b)} = 0\) \(\forall n = 0,1,2,\dots\).

Next, note that \(<y \Pnkab, P_{m,j}^{(a,b)}>_\Wab = 0\) for \(m < n-1\) (and for \(m > n+2\)). Thus for \(m = n-1, n, n+1,\) \(j = 0,\dots,m:\)
\begin{align}
<y \Pnkab, P_{m,j}^{(a,b)}>_\Wab &= \int \int_\Omega \Pnkab(x,y) \: \Pmjab(x,y) \: y \: \Wab(x,y) \: dy \: dx \\
&= \Big( \int^1_0 H_{n-k}^{(a, b+k+\half)}(s) \: H_{m-j}^{(a, b+j+\half)}(s) \: s^a \: \rho(s)^{2b+k+j+2} \: ds \Big) \nonumber \\
& \quad \quad \quad\cdot \: \Big( \int^1_{-1} P_k^{(b,b)}(t) \: P_j^{(b,b)}(t) \: t \: (1-t^2)^b \: dt \Big) \\
&= \begin{cases}
    	\delta_k^{(b)} \: <H_{n-k}^{(a, b+k+\half)}, H_{m-k-1}^{(a, b+k+\frac{3}{2})}>_{H^{(a, b+k+\frac{3}{2})}} \quad& \text{if } j = k+1 \\
	\delta_{k-1}^{(b)} \: <H_{n-k}^{(a, b+k+\half)}, H_{m-k+1}^{(a, b+k-\half)}>_{H^{(a, b+k+\half)}} \quad& \text{if } j = k-1 \\
	0 & \text{o/w}
      \end{cases}.
\end{align}

We then have that multiplication by \(x\) and \(y\) yields the 3-term recurrences as follows:
\begin{align}
x \Pnk^{(a,b)}(x,y) &= \alphaab_{n,k,1} \: P_{n-1, k}^{(a,b)}(x, y) + \alphaab_{n,k,2} \: P_{n, k}^{(a,b)}(x, y) + \alphaab_{n+1,k,1} \: P_{n+1, k}^{(a,b)}(x, y), \\
y \Pnk^{(a,b)}(x,y) &= \betaab_{n,k,1} \: P_{n-1, k-1}^{(a,b)}(x, y) + \betaab_{n,k,2} \: P_{n-1, k+1}^{(a,b)}(x, y) \nonumber \\
		& \quad \quad + \betaab_{n,k,3} \: P_{n, k-1}^{(a,b)}(x, y) + \betaab_{n,k,4} \: P_{n, k+1}^{(a,b)}(x, y) \nonumber \\
		& \quad \quad + \betaab_{n,k,5} \: P_{n+1, k-1}^{(a,b)}(x, y) + \betaab_{n,k,6} \: P_{n+1, k+1}^{(a,b)}(x, y),
\end{align}
for \((x,y) \in \Omega\), where
\begin{align}
\alphaab_{n,k,1} &:= \beta_{n-k-1}^{(a, b+k+\half)} \\
\alphaab_{n,k,2} &:= \alpha_{n-k}^{(a, b+k+\half)} \\
\betaab_{n,k,1} &:= \frac{\delta_{k-1}^{(b)}}{|| P_{n-1,k-1}^{(a,b)} ||_{W^{(a,b)}}^2} \: <H_{n-k}^{(a, b+k+\half)}, H_{n-k}^{(a, b+k-\half)}>_{H^{(a, b+k+\half)}} \\
\betaab_{n,k,2} &:= \frac{\delta_{k}^{(b)}}{|| P_{n-1,k+1}^{(a,b)} ||_{W^{(a,b)}}^2} \: <H_{n-k}^{(a, b+k+\half)}, H_{n-k-2}^{(a, b+k+\frac{3}{2})}>_{H^{(a, b+k+\frac{3}{2})}} \\
\betaab_{n,k,3} &:= \frac{\delta_{k-1}^{(b)}}{|| P_{n,k-1}^{(a,b)} ||_{W^{(a,b)}}^2} \: <H_{n-k}^{(a, b+k+\half)}, H_{n-k+1}^{(a, b+k-\half)}>_{H^{(a, b+k+\half)}} \\
\betaab_{n,k,4} &:= \frac{\delta_{k}^{(b)}}{|| P_{n,k+1}^{(a,b)} ||_{W^{(a,b)}}^2} \: <H_{n-k}^{(a, b+k+\half)}, H_{n-k-1}^{(a, b+k+\frac{3}{2})}>_{H^{(a, b+k+\frac{3}{2})}} \\
\betaab_{n,k,5} &:= \frac{\delta_{k-1}^{(b)}}{|| P_{n+1,k-1}^{(a,b)} ||_{W^{(a,b)}}^2} \: <H_{n-k}^{(a, b+k+\half)}, H_{n-k+2}^{(a, b+k-\half)}>_{H^{(a, b+k+\half)}} \\
\betaab_{n,k,6} &:= \frac{\delta_{k}^{(b)}}{|| P_{n+1,k+1}^{(a,b)} ||_{W^{(a,b)}}^2} \: <H_{n-k}^{(a, b+k+\half)}, H_{n-k}^{(a, b+k+\frac{3}{2})}>_{H^{(a, b+k+\frac{3}{2})}}. 
\end{align}

Now, set \(J_x^{(a,b)}, J_y^{(a,b)}\) as the \(\half (N+1)(N+2) \times \half (N+1)(N+2)\) Jacobi matrices corresponding to
\begin{align}
J_x^{(a,b)} \bigPab(x,y) = x \bigPab(x,y), \quad J_y^{(a,b)} \bigPab(x,y) = y \bigPab(x,y).
\end{align}
Then  \(J_x^{(a,b)}, J_y^{(a,b)}\) have the Banded-Block-Banded structures
\begin{align}
J_{x/y}^{(a,b)} &= \begin{bmatrix}
		B^{x/y}_0 & A^{x/y}_0 & & & & \\
		C^{x/y}_1 & B^{x/y}_1 & A^{x/y}_1 & & & \\
		& C^{x/y}_2 & B^{x/y}_2 & A^{x/y}_2  & & & \\
		& & C^{x/y}_3 & \ddots & \ddots & \\
		& & & \ddots & \ddots & \ddots \\
	\end{bmatrix}
\end{align}
where
\begin{align}
A^x_n &:= \begin{bmatrix}
		\alphaab_{n+1,0,1} & 0 & \hdots & 0 \\
		& \ddots & & \vdots & \\
		& & \alphaab_{n+1,n,1} & 0 \\
	    \end{bmatrix} \in \R^{(n+1)\times(n+2)}, \quad n = 0,1,2,\dots,N-1 \\
B^x_n &:= \begin{bmatrix}
		\alphaab_{n,0,2} & & \\
		& \ddots & \\
		& & \alphaab_{n,n,2} \\
	    \end{bmatrix} \in \R^{(n+1)\times(n+1)} \quad n = 0,1,2,\dots,N \\
C^x_n &:= \big( A^x_n \big)^T \in \R^{(n+1)\times n},  \quad n = 1,2,\dots,N, \\ 
\nonumber \\
A^y_n &:= \begin{bmatrix}
		0 & \betaab_{n,0,6} & & & \\
		\betaab_{n,1,5} & 0 & & & \\
		& \ddots & \ddots & \ddots & \\
		& & \betaab_{n,n,5}& 0 & \betaab_{n,n,6} \\
	    \end{bmatrix} \in \R^{(n+1)\times(n+2)}, \quad n = 0,1,2,\dots,N-1 \\
B^y_n &:= \begin{bmatrix}
		0 & \betaab_{n,0,4} & & \\
		\betaab_{n,1,3} & 0 & \ddots & \\
		& \ddots & \ddots & \betaab_{n,n-1,4} \\
		& & \betaab_{n,n,3} & 0
	    \end{bmatrix} \in \R^{(n+1)\times(n+1)}  \quad n = 0,1,2,\dots,N \\
C^y_n &:= \begin{bmatrix}
		0 & \betaab_{n,0,2} & & \\
		\betaab_{n,1,1} & 0 & \ddots & \\
		& \ddots & \ddots & \betaab_{n,n-2,2} \\
		& & \ddots & 0 \\
		& & & \betaab_{n,n,1} \\
	    \end{bmatrix} \in \R^{(n+1)\times n}, \quad n = 1,2,\dots,N.
\end{align}

\subsection{Clenshaw on the half disk}

Combining each system in (71) we can write the block-tridiagonal system
\begin{align}
\renewcommand\arraystretch{1.3}
\mleft[
\begin{array}{cccc}
		1 & & & \\
		B_0-G_0(x,y) & A_0 & & \\
		C_1 & B_1-G_1(x,y) & \quad A_1 \quad & \\
		& C_2 & \ddots & \ddots \\
		& & \ddots &
\end{array}
\mright]
\bigPab(x,y)
=
\begin{bmatrix}
	P^{(a,b)}_{0,0} \\ 0 \\ 0 \\ \vdots \\ 0 \\
\end{bmatrix}.
\end{align}
where we note \(P^{(a,b)}_{0,0}(x,y) \equiv H_0^{(a,b)} \: P_0^{(b,b)}\), and for each \(n = 0,1,\dots,N\):
\begin{align}
A_n &:= \begin{bmatrix}
		A^x_n \\
		A^y_n \\
		A^z_n
	    \end{bmatrix} \in \R^{2(n+1)\times(n+3)} \quad (n \ne N), \quad
C_n := \begin{bmatrix}
		C^x_n \\
		C^y_n
	    \end{bmatrix} \in \R^{2(n+1)\times n} \quad (n \ne 0), \nonumber \\
B_n &:= \begin{bmatrix}
		B^x_n \\
		B^y_n
	    \end{bmatrix} \in \R^{2(n+1)\times(n+1)}, \quad
G_n(x,y,z) := \begin{bmatrix}
		xI_{n+1} \\
		yI_{n+1}
	    \end{bmatrix} \in \R^{2(n+1)\times(n+1)}.
\end{align}
 
For each \(n = 0,1,\dots,N-1\) let \(\Dnt\) be any matrix that is a left inverse of \(A_n\), i.e. such that \(\Dnt A_n = I_{n+3}\). Multiplying our system by the preconditioner matrix that is given by the block diagonal matrix of the \(\Dnt\)'s, we obtain a lower triangular system \citep{dunkl2014orthogonal} (p78), which can be expanded to obtain the recurrence:
\begin{align}
\begin{cases}
\bigPab_{-1}(x,y) := 0 \\
\bigPab_{0}(x,y) := P^{(a,b)}_{0,0} \\
\bigPab_{n+1}(x,y) = -\Dnt (B_n-G_n(x,y)) \bigPab_n - \Dnt C_n  \,\bigPab_{n-1}(x,y), \quad n = 1,2,3,\dots,N.
\end{cases}
\end{align}

Note that we can define \(\Dnt\) as follows. For \(n\) even:
\begin{align}
\Dnt := \begin{bmatrix}
		\frac{1}{\alphaab_{n+1,0,1}} & & &  \\
		& \ddots & & & \\
		& & \frac{1}{\alphaab_{n+1,n,1}} & \\
		0 & \hdots & 0 & \eta_{m-1} & \hdots & 0 & \eta_1 & 0 & \eta_0
	    \end{bmatrix},
\end{align}
where
\begin{align}
m &= \frac{n}{2} + 1, \quad \eta_0 = \frac{1}{\betaab_{n,n,6}}, \nonumber \\
\eta_k &= -\frac{\betaab_{n,n-2(k-1),5} \: \eta_{k-1}}{\betaab_{n,n-2k,6}} \quad k = 1,\dots,m-1.
\end{align}

For \(n\) odd:

\begin{align}
\Dnt := \begin{bmatrix}
		\frac{1}{\alphaab_{n+1,0,1}} & &  \\
		& \ddots & & &  \\
		& & \ddots & & \\
		& & & \frac{1}{\alphaab_{n+1,n,1}} & \\
		\xi & 0 & \hdots & 0 & 0 & \eta_{m-2} & \hdots & 0 & \eta_0
	    \end{bmatrix},
\end{align}
 where
\begin{align}
m &= \frac{n+1}{2} + 1, \quad \eta_0 = \frac{1}{\betaab_{n,n,6}}, \nonumber \\
\eta_k &= -\frac{\betaab_{n,n-2(k-1),5} \: \eta_{k-1}}{\betaab_{n,n-2k,6}} \quad k = 1,\dots,m-2, \\
\xi &= -\frac{\betaab_{n,1,5} \: \eta_{m-2}}{\alphaab_{n+1,0,1}}. \nonumber
\end{align}

\subsection{Example: Laplace operator}

We can obtain the matrix operator for \(\Delta\), that will take us from coefficients for expansion in the weighted space
\[
\Wii(x,y) \: \bigP^{(1,1)}(x,y)
\]
to coefficients in the non-weighted space
\[
\bigP^{(1,1)}(x,y).
\]
Importantly, this operator will have Banded-Block-Banded structure, and hence will be sparse. Note that this construction will ensure the imposition of the Dirichlet zero boundary conditions on $\Omega$.

For general \(a,b\) and general function \(f(x,y) = \mathbf{f}^T \: \bigPab(x,y)\), there exist Banded-Block-Banded matrix operators \(D_x^{(a,b)}, \: D_y^{(a,b)}, \: W_x^{(a,b)}, \: W_y^{(a,b)}\) such that:
\begin{align}
\dx[f(x,y)] &= \bigP^{(a+1,b+1)}(x,y)^T \: D_x^{(a,b)} \: \mathbf{f}, \\
\dy[f(x,y)] &= \bigP^{(a,b+1)}(x,y)^T \: D_y^{(a,b)} \: \mathbf{f}, \\
\dx[\Wab(x,y) \: f(x,y)] &= W^{(a-1,b-1)}(x,y) \: \bigP^{(a-1,b-1)}(x,y)^T \: W_x^{(a,b)} \: \mathbf{f}, \\
\dy[\Wab(x,y) \: f(x,y)] &= W^{(a,b-1)}(x,y) \: \bigP^{(a,b-1)}(x,y)^T \: W_y^{(a,b)} \: \mathbf{f}.
\end{align}
The incrementing and decrementing of parameters as seen here is analogous to other well known orthogonal polynomials families' derivatives, for example the Jacobi polynomials, as seen in the DLMF \cite{DLMFDerivatives}.
Further, we note that there exist Banded-Block-Banded matrix operators \(T^{(1,0)\to(0,0)}\) and \(T^{(0,1)\to(1,1)}\) such that:
\begin{align}
x \bigP^{(1,0)}(x,y) &= \Big(T^{(1,0)\to(0,0)} \Big)^T \: \bigP^{(0,0)}(x,y) \\
\bigP^{(0,1)}(x,y) &= \Big(T^{(0,1)\to(1,1)} \Big)^T \: \bigP^{(1,1)}(x,y)
\end{align}
Then, the operator \(\Delta\) is given by
\begin{align}
    \Delta := D_x^{(0,0)} \: W_x^{(1,1)} + T^{(0,1)\to(1,1)} \: D_y^{(0,0)} \: T^{(1,0)\to(0,0)} \: W_y^{(1,1)}
\end{align}

An example is as follows. Fix \(a, b = 1,1\). Find \(u(x,y)\) given a function \(f(x,y)\) such that:
\begin{align}
    \Delta(\Wii \: u)(x,y) = f(x,y) \quad \text{in } \Omega
    \end{align}
    where we recall
    \begin{align}
    \Wii(x,y) = x \: (1-x^2-y^2)
\end{align}
which ensures a boundary condition that \((\Wii \: u)(x,y) = 0\) on \(\partial \Omega\).

The first step is to obtain the expansion of \(f\) in the \(\Pii\) basis, i.e. obtain the vector \(\mathbf{f}\) such that \(f(x,y) = \mathbf{f}^T \: \bigP^{(1,1)}(x,y) = \sum_{n=0}^N \mathbf{f}_n^T \: \bigP_n^{(1,1)}(x,y)\) for some \(N \in \N\). The coefficients of \(u\) in the \(\Pii\) basis are then given by 
\begin{align}
    \mathbf{u} = \Delta \text{\textbackslash} \mathbf{f}.
\end{align}



\section{Future work - orthogonal polynomials on the hemisphere and spherical triangles}

We will use a similar idea of projecting up from 1D OPs to the interval, to 2D OPs on the disk, to 3D OPs on the sphere. Initially, we will add to the ApproxFun library by adding capability to work with OPs on the half disk (2D) with weights governed by parameters that will be incremented/decremented after an application of an operator (e.g. partial derivative). With this, we can obtain a family of OPs on the hemisphere, with sparse relations.

Beyond this, solving PDEs on isosceles triangles should follow using the same procedure. However, we are expecting that to solve PDEs on spherical triangles (as is our ultimate goal) will require further ideas beyond the procedures given in this document.


\bibliography{esa-bib}


\end{document}











  