
\chapter{Introduction}\label{CHAPTER:intro}

\bstodoinline{Any comments in boxes like these indicate changes made.

Aside from where I have marked, I have also:

- Fixed typos pointed out and found myself

- Added commas or full stops after equations

- Changed how commas are presented in equations with \enquote{cases}

- Added round brackets to citations of equation numbers in the DLMF

- Added a dot where citing a page number (eg, p.76)

- Replaced \& with and}

Univariate orthogonal polynomials (hereon also referred to as OPs) have been extensively involved in the development of multiple fields of computational and applied mathematics. For example, univariate OPs have been used to derive spectral methods to numerically solve one-dimensional differential equations (see e.g. \cite{trefethen2000spectral, canuto2007spectral, gottlieb1977numerical, boyd2001chebyshev, mason2002chebyshev, shen2011spectral, olver2013fast}). While there are many famous examples of univariate OPs -- such as the Jacobi polynomials, with special cases of the Legendre polynomials and the Chebyshev polynomials but to name a few of the classical families \cite[\S18.3]{DLMF}\bstodo{Clarified that these families named are all Jacobi} -- the area of multivariate orthogonal polynomials has a smaller array of research. 

One could say this is surprising, given that there is a long history of around 150 years associated with multivariate OPs, beginning with Hermite first presenting the multivariate Hermite polynomials in 1865 \cite{appel1926fonctions, ismail2017review}. Zernike polynomials \cite{zernike1934diffraction} were first introduced in 1934, as another example, that are a group of bivariate polynomials orthogonal on a unit circle. Koornwinder in 1975 described a method for constructing two-variable OPs from univariate OPs \cite{koornwinder1975two}. However, few books have been published over the years on the topic of multivariate orthogonal polynomials, with the notable exception of the book by Dunkl and Xu \cite{dunkl2014orthogonal}. As Dumitriu, Edelman and Shuman said in 2007, \enquote{[multivariate orthogonal polynomials] are understudied, underapplied, and important applications may be being missed} \cite{dumitriu2007mops}, while in 2016, Olver opined that \enquote{there is still a long way to go before multivariate orthogonal polynomials can reach their full potential in applications}. He hypothesised that \enquote{[p]art of this neglect may be due to the fact that mutually orthogonal polynomials are not uniquely defined: there is no canonical ordering} and continued \enquote{[e]ven when explicit constructions are known, these are often unwieldy} \cite{olver2016review}.

This branch of mathematics has an encouraging future though, not least as a basis for sparse spectral methods for solving partial differential equations (PDEs) on multidimensional domains. Spectral methods have been developed for solving PDEs as an alternative to finite difference and finite element methods. For example, in recent years spectral methods have been established on the triangle \cite{olver2019triangle} using OPs inspired by the Koornwinder approach for defining them, and on the disk \cite{vasil2016tensor} using Zernike polynomials.

There are various interpretations of what a \enquote{spectral method} is. It could be described as one that achieves spectral convergence of its solutions \cite{gottlieb1977numerical}, or one that uses Laplacian eigenfunctions as basis functions \cite{zhong2007numerical}, or one that uses OPs as basis functions for the approximation of a solution \cite{olver2019triangle}. It is the latter that we refer to for our purposes as a spectral method in the body of the thesis. By utilising OPs as basis functions, we can develop sparse spectral methods (SSMs), meaning that the naturally sparse relationships between the basis OPs lead to sparse operator matrices that represent the differential operations in the equation to solve. Sparse spectral methods for one dimensional problems have been shown to lead to \enquote{almost banded} matrices \cite{olver2013fast}.

In this thesis we expand upon this knowledge by providing frameworks for similar sparse spectral methods for solving PDEs on other multidimensional domains (notably including the disk-slice and trapezium in 2D, and the spherical cap surface in 3D) that also yield matrices that are what is defined as \enquote{banded-block-banded}. We take inspiration from the work established for the triangle \cite{olver2019triangle} and the unit disk \cite{vasil2016tensor}, which can  be seen as special cases of the disk-slice and trapezium in the framework we present here. The spherical cap work serves to lay a foundation for using spherical caps and spherical bands as elements in a spectral element method for solving PDEs on the whole sphere, as an alternative to the spherical harmonic transform approach that the European Centre for Medium-range Weather Forecasts (ECMWF) use in their weather and climate model \cite{cheong2006dynamical}. 

Spherical harmonics (SHs) are of course a long-established and famous group of functions defined on the surface of a sphere with certain useful properties (for example, they are orthogonal to each other on the unit sphere, have sparse recurrences, and are eigenfunctions of the spherical Laplacian \bstodo{Re-worded the properties}) and as a result are widely used in many scientific fields for solving PDEs including computer graphics (e.g. \cite{moon2008efficient, sloan2013efficient}), astrophysics (e.g. \cite{vasil2019tensor}), quantum theory (e.g. \cite{varshalovich1988quantum}), biochemistry (e.g.\cite{parimal2014application, basko1998application}), geosciences (e.g. \cite{fletcher2017data, hollerbach2013parity}) and meteorology (e.g. \cite{evans1998spherical, rubinstein2015scalar, wedi2013fast, ecmwf2020scalability, courtier1998ecmwf, silberman1954planetary, gottlieb1977numerical}). Spectral methods on the sphere involving spherical harmonics have been used for over 60 years \cite{silberman1954planetary}. Notably, spherical harmonics are also used as basis functions for the spectral transform method that makes up part of the model in the Integrated Forecasting System (IFS), which is used by ECMWF for their forecasts \cite{wedi2013fast}. 

There are other modelling systems using global spectral models too. For example, the AGCM3 model developed by the Canadian Centre for Climate Modelling and Analysis (CCCma) and the Community Earth System Model (CESM) both use a spectral dynamical core\footnote{A \textit{dynamical core} is part of a model that deals with numerically solving the equations of motion on the underlying grid, as opposed to other physical processes.} \cite{hurrell2013community, scinocca2008cccma}, while the Geophysical Fluid Dynamics Laboratory (GFDL) has also developed a global spectral atmospheric model \cite{gordon1982description}.

While the whole sphere spectral method using the spherical harmonics has been successful for numerous years \cite{williamson2007evolution}, there is a drawback in the parallel scalability bottleneck that arises from the global spectral transform, which is expected to inhibit future performance of the IFS \cite{ecmwf2020scalability, wedi2013fast}. 

Many implementations of an algorithm to compute the spectral transform (or spherical harmonic transform) exist (see e.g. \cite{slevinsky2019fast, suda2002fast}). For the IFS, the spherical harmonic transform in fact uses two transforms -- a Fourier transform (using the well-established Fast Fourier Transform (FFT) \cite{cooley1965algorithm}) in the longitudinal direction and a Legendre transform in the latitudinal direction -- and it is the Legendre transform that has been identified as inhibiting future performance due to its computational cost. While a Fast Legendre Transform (FLT) \cite{wedi2013fast} has been incorporated into the model, along with new grid types \cite{malardel2016new}, to help to extend the lifespan of the spectral method for numerical weather prediction (NWP), it may not be sufficient for certain desired cases and resolutions \cite{wedi2014increasing}. 

The motivation for this project was to help address this problem while still utilising a spectral approach. More precisely, we aim to develop a sparse spectral method for solving PDEs on the spherical cap as a surface in 3D, with a simple extension to a spherical band. Together, these frameworks can be pieced together to create a spectral element method for the whole sphere, or further developed to investigate spectral methods on other spherical subdomains. This, however, is future work beyond the scope of this thesis. By spectral element method, we mean a finite element method (FEM) that uses high degree basis polynomials for its elements (this could also be referred to as a $p$-FEM with large $p$). In other words, we can use our spectral methods developed for the spherical band and cap elements as part of a finite element framework. By using this approach, one can avoid having to compute the global spectral transform (in particular, the global Legendre transform) and instead simply apply the local element transforms in parallel. Moreover, by still using a spectral approach, one can maintain the high accuracy and excellent error properties that such methods bring. Solving PDEs on a section of the sphere surface is still useful for numerical weather prediction in its own right too, where one could for example perform a more localised simulation near a pole while still using a \enquote{global} (in the domain sense) spectral method approach. Moreover, there are also prospective applications in physics, particularly in astrophysics, where solving PDEs on the sphere surface and working in spherical geometries is also desirable (e.g. \cite{vasil2019tensor, reinecke2013libsharp, beyer2014numerical, varshalovich1988quantum, slevinsky2018spectral, rubinstein2015scalar}).

Recently, a method for computing tensor fields in spherical coordinates using Jacobi polynomials has been proposed \cite{vasil2019tensor}. In this work the authors present the method for both the surface of the unit sphere and the three-dimensional generalisation of the unit ball. Their method involves using a spectral basis to represent functions too, choosing for the angular part of the basis to be the spin-weighted spherical harmonics (of which the spherical harmonics are a special case, with spin $0$) in spherical coordinates. By using these as their basis functions, they are able to derive sparse relations for the Laplacian operator, as well as for multiplication of the basis by $\cosphi$ and $\sinphi$ (where $\varphi$ is the polar angle from the $z$ axis in cartesian coordinates), which lead to operators for operations with angular dependencies involving these. 

On the other hand, we aim to propose a strictly orthogonal \textit{polynomial} basis, and derive sparse operators for multiplication by the cartesian coordinate axes, which can in turn lead to operators for multiplication by trigonometric functions too. By utilising orthogonal polynomials in cartesian coordinates, we can develop a sparse spectral method on subdomains of the sphere surface (e.g. the spherical cap) using our knowledge of other geometries and methods involving OPs developed for them.


% GENERAL FRAMEWORK
\bstodoinline{The following is the added "roadmap"}
The construction of the spectral methods in this thesis follow a similar pattern that is worth detailing here as a general framework.
\begin{enumerate}
	\item \textit{Define the domain of interest.}
	
		It is best to do this via a restriction of one coordinate in terms of the others, using a function $\rho$ that satisfies one of two conditions we will detail later. As an example, for a domain in 2D space, we would define this as $\Omega := \bsset{(x,y) \in \R^2}{\alpha \le x \le \beta, \: \gamma \rho(x) \le y \le \delta \rho(x)}$ for given constants $\alpha, \beta, \gamma, \delta$.
	
	\item \textit{Express the multivariate and multi-parameter OPs as polynomials in cartesian coordinates.}
	
		The construction of the multivariate OPs takes its inspiration from the Koornwinder approach \cite{koornwinder1975two}. The construction will make use of the function $\rho$. These OPs will be $d$-parameter, meaning that we can construct operators that raise or lower the parameter values, and thus represent conversion between the OP bases.
	
	\item \textit{Derive expressions for multiplication of the OPs by the coordinates.}
	
	\item \textit{Use these relations to determine the entries to \enquote{Jacobi matrices}}
	
		These \enquote{Jacobi matrices} $J_x$, $J_y$, etc. represent multiplication by the coordinates when applied to the degree-ordered vector of OPs, $\bigP$. They will be banded-block-banded in structure -- in fact, due to the construction, they will be block-tridiagonal. Continuing with the 2D example:
		\bseqn{
			J_{x/y} &= 
			\begin{pmatrix}
				B_{x/y,0} & A_{x/y,0} & & & & \\
				C_{x/y,1} & B_{x/y,1} & A_{x/y,1} & & & \\
				& C_{x/y,2} & B_{x/y,2} & A_{x/y,2}  & & & \\
				& & C_{x/y,3} & \ddots & \ddots & \\
				& & & \ddots & \ddots & \ddots \\
			\end{pmatrix}.
		}
		The degree-ordered vector of OPs, $\bigP$, can be thought of as being made up of ordered sub-vectors grouped by polynomial degree, $\bigP_n$ for $n \in \No$.
		
	\item \textit{Create a three-term recurrence for the vector of OPs of degree $n$.}
	
		This is done by combining the systems $J_x \bigP = x \bigP$ and $J_y \bigP = y \bigP$ (and $J_z \bigP = z \bigP$ if in three-dimensional space). The resulting three-term recurrence will have the form:
		\bseqn{
			\bigP_{n+1}(x,y) = -\Dnt (B_n-G_n(x,y)) \bigP_n(x,y) - \Dnt C_n  \, \bigP_{n-1}(x,y).
		}
		Here, the matrices $B_n$, $C_n$ and $G_n$ are defined via:
		\bseqn{
                    	C_n := 
                    		\begin{pmatrix}
                    			C_{x,n} \\
	                    		C_{y,n}
				\end{pmatrix} \: (n \ne 0), \quad
			B_n &:= 
				\begin{pmatrix}
                    			B_{x,n} \\
	                    		B_{y,n}
				\end{pmatrix}, \quad
			G_n(x,y) := 
                    		\begin{pmatrix}
					xI_{n+1} \\
	                    		yI_{n+1}
				\end{pmatrix}.
		}
		The matrices $\Dnt$ are left-inverses of the matrices $A_n$ respectively, which are defined similarly:
		\bseqn{		
			A_n &:=
				\begin{pmatrix}
                    			A_{x,n} \\
                    			A_{y,n}
				\end{pmatrix}.
		}
		
	\item \textit{Use the multivariate version of the Clenshaw algorithm for evaluation of functions that are expanded in the OP basis.}
	
	\item \textit{Derive (sparse) relationships for various differential operations, and relations for raising and lowering the parameter values of the OPs.}
	
	\item \textit{Construct the sparse operator matrices representing these operations.}
		
		The differential operators may raise or lower parameter values, which is why the basis conversion operators are required. Further operators can be constructed as combinations of others -- for example, a second derivative would be the multiplication of two first derivative operators. Notably, these operator matrices will be banded-block-banded in structure.
		
	\item \textit{Use the \enquote{operator Clenshaw algorithm} to construct operator matrices for the action of multiplying by a general function.}
	
		For instance, this would be useful for constructing an operator matrix that represents a variable coefficient in a Helmholtz example.
\end{enumerate}

By following this framework, one will be able to write down a given PDE in spectral space (i.e. as a matrix-vector equation). The variables are given by their coefficients for their expansion in the OP basis, while the differential and other operations are given by their relevant operator matrices. The resulting system's sparsity is a result of the sparse relations of the OPs.
\bstodoinline{End of added roadmap}


The structure of the remainder of thesis is as follows.

\bsrefchapter{CHAPTER:sphericalharmonics} of this thesis provides an introduction to sparse spectral methods via the spherical harmonics on the whole sphere surface. Here, we think of the spherical harmonics in a non-traditional way and write them as a group of multidimensional orthogonal polynomials in $(x,y,z)$ as opposed to functions of spherical coordinates. We present an OP framework for how the spherical harmonics can be used to expand functions defined on the sphere as multidimensional polynomials in $x, y, z$, and how differential operators can be applied as banded-block-banded matrix operators to coefficient vectors for a function's expansion. Further, we demonstrate how the vector spherical harmonics (VSHs) can be used as an orthogonal basis for vector valued functions lying in the \enquote{tangent bundle} of the sphere, and thus how one can additionally derive gradient and divergence operators.

In \bsrefchapter{CHAPTER:diskslice} we move on to working in 2D, where in recent years sparse spectral methods for solving PDEs have been derived using hierarchies of classical orthogonal polynomials on intervals, disks, and triangles. Presenting a new framework for choosing a suitable orthogonal polynomial basis for more general 2D domains defined via an algebraic curve as a boundary, this work builds on the observation that sparsity is guaranteed due to this definition of the boundary, and that the entries of partial differential operators can be determined using formulae in terms of (non-classical) univariate orthogonal polynomials, which we define. Triangles and the full disk are then special cases of our framework, which we formalise for the disk-slice and trapezium cases. The work in this chapter has been previously published \cite{snowball2019sparse}.

With a greater knowledge base in our quiver, we can adapt the techniques learnt from the founding of the disk-slice formulation to surfaces in 3D in \bsrefchapter{CHAPTER:sphericalcaps}. Using the same family of (non-classical) 1D OPs, we present a suitable orthogonal polynomial basis for the spherical cap, a subdomain of the surface of a unit sphere, complete with sparse differential operators. A relatively simple adaption permits this framework to be extended to a spherical band. From here, a spectral element method could be devised for the whole sphere using the aforementioned as elements. The work in this chapter has been accepted for publication \cite{snowball2020sparse}.

Chapter 5 gives a summary of the work we have presented and details avenues for future directions that one could take it.

Finally, as an appendix, we outline how one could approach a vector OP basis for the \enquote{tangent bundle} of the spherical cap, analogous to the vector spherical harmonics on the whole sphere. This constitutes \bsrefappendix{appendix:tangent}.
