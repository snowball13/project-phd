
\chapter{Introduction}

Univariate orthogonal polynomials (hereon also referred to as OPs) have been extensively involved in the development of multiple fields of computational and applied mathematics \cite{onedimopsrefs} \bstodo{citation here}. For example, univariate OPs have been used to derive spectral methods to numerically solve one-dimensional differential equations \cite{boyd2001chebyshev, olver2013fast}. There are many famous examples of univariate OPs -- such as the Jacobi polynomials, the Legendre polynomials and the Chebyshev polynomials but to name a few \cite[Section 18.3]{DLMF} -- but the area of multivariate orthogonal polynomials has a smaller array of research. However, this branch of mathematics has an encouraging future \cite{dunkl2014orthogonal}, not least as a basis for sparse spectral methods for solving partial differential equations (PDEs) on multidimensional domains, such as on the triangle \cite{olver2019triangle} with applications including solving Volterra integral equations \cite{gutleb2020sparse}. 

For the uninitiated, a spectral method is one that, given a domain, numerically approximates the solution of a (differential) equation as sum of basis functions for said domain, with the coefficients to be found. The basis functions are non-zero over the whole domain, meaning the solution is found globally and thus helps such methods to be inherently accurate (providing the solution is smooth). By utilising OPs as basis functions, we can develop sparse spectral methods (SSMs), meaning that the naturally sparse relationships between the basis OPs leads to sparse operator matrices that represent the differential operations in the equation to solve. Sparse spectral methods for one dimensional problems have been shown to lead to \enquote{almost banded} matrices \cite{olver2013fast}. 

In this thesis we expand upon this knowledge by providing frameworks for similar sparse spectral methods for solving PDEs on other multidimensional domains (notably including the disk-slice and trapezium in 2D, and the spherical cap surface in 3D) that also yield matrices that are similarly \enquote{almost banded}. The spherical cap work serves to lay a foundation for using spherical caps and spherical bands as elements in a spectral element method for solving PDEs on the whole sphere, as an alternative to the Spherical Harmonic approach that the European Centre for Medium-range Weather Forecasts (ECMWF) use in their weather and climate model \cite{cheong2006dynamical}.

Spherical Harmonics are of course one long-established and famous group of multidimensional orthogonal polynomials. They are used as basis functions for the spectral transform method that makes up part of the model in the Integrated Forecasting System (IFS), which is used by ECMWF for their forecasts \cite{wedi2013fast}. While the whole sphere spectral method using the Spherical Harmonics has been successful for numerous years \cite{williamson2007evolution}, there is a drawback in the parallel scalability bottleneck that arises from the global spectral transform, which is expected to inhibit future performance of the IFS \cite{ecmwf2020scalability, wedi2013fast, diamantakisecmwf}. 

This spherical harmonic transform in fact uses two transforms -- a Fourier transform (using the well established Fast Fourier Transform \cite{cooley1965algorithm}) in the longitudinal direction and a Legendre transform in the latitudinal direction -- and it is the Legendre transform that has been identified as inhibiting future performance due to its computational cost. While a Fast Legendre Transform (FLT) \cite{wedi2013fast} has been incorporated into the model, along with new grid types \cite{malardel2016new}, to help to extend the lifespan of the spectral method for numerical weather prediction (NWP), it may not be sufficient for certain desired cases and resolutions \cite{diamantakisecmwf}. 

The motivation for this project was to help address this problem while still utilising a spectral approach. More precisely, we aim to develop a sparse spectral method for solving PDEs on the spherical cap as a surface in 3D, with a simple extension to a spherical band. Together, these frameworks can be pieced together to create a spectral element method for the whole sphere, or further developed to investigate spectral methods on other spherical subdomains. By spectral element method, we mean a finite element method (FEM) that uses high degree basis polynomials for its elements (this could also be referred to as a $p$-FEM with large $p$). In other words, we can use our spectral methods developed for the spherical band and cap elements as part of a finite element framework. By using this approach, one would be able to avoid having to complete the global spectral transform (in particular, the global Legendre transform) and instead be able to simply apply the local element transforms in parallel. Moreover, by still using a spectral approach, one can maintain the high accuracy and excellent error properties that such methods bring.

Chapter 2 of this thesis provides an introduction to sparse spectral methods via the Spherical Harmonics on the whole sphere surface. We present a framework for how the Spherical Harmonics can be used to expand functions defined on the sphere as multidimensional polynomials in $x, y, z$, and how differential operators can be applied as banded-block-banded matrix operators to coefficient vectors for the function's expansion. Further, we demonstrate how the Vector Spherical Harmonics can be used as an orthogonal basis for vector valued functions lying in the tangent space of the sphere, and thus how one can additionally derive gradient and divergence operators.

In Chapter 3 we move on to working in 2D, where in recent years sparse spectral methods for solving PDEs have been derived using hierarchies of classical orthogonal polynomials on intervals, disks, and triangles. Presenting a new framework for choosing a suitable orthogonal polynomial basis for more general 2D domains defined via an algebraic curve as a boundary, this work builds on the observation that sparsity is guaranteed due to this definition of the boundary, and that the entries of partial differential operators can be determined using formulae in terms of (non-classical) univariate orthogonal polynomials, which we define. Triangles and the full disk are then special cases of our framework, which we formalise for the disk-slice and trapezium cases.

With a greater knowledge base in our quiver, we can adapt the techniques learnt from the founding of the disk-slice formulation to surfaces in 3D in Chapter 4. Using the same family of (non-classical) 1D OPs, we present a suitable orthogonal polynomial basis for the spherical cap, a subdomain of the surface of a unit sphere, complete with sparse differential operators. A relatively simple adaption permits this framework to be extended to a spherical band. From here, a spectral element method could be devised for the whole sphere using the aforementioned as elements.

