
\addcontentsline{toc}{chapter}{Abstract}

\begin{abstract}

%This thesis develops sparse spectral methods for solving partial differential equations (PDEs) on various multidimensional domains, with a specific focus on the disk-slice and trapezium in 2D, and the spherical cap as a surface in 3D. For the latter scenario, the PDEs are surface PDEs involving Laplace-Beltrami operators, spherical gradients and other spherical operators.  
%
%Throughout this thesis we wish to utilise our understanding of the orthogonal polynomials and sparse spectral methods to build up to new methods for subdomains of the sphere surface. To this end, we begin with an introduction to sparse spectral methods via the long established Spherical Harmonics on the whole sphere surface. We present a framework for how the spherical harmonics can be thought of as a set of multidimensional orthogonal polynomials in $x$, $y$ and $z$, and can be used to expand functions defined on the sphere. We explain how differential operators can be applied as banded-block-banded matrix operators to coefficient vectors for the function's expansion. This way of thinking about the spherical harmonics differs from the traditional sense (where they are considered as orthogonal functions in spherical coordinates) but we do so as to be able to generalise the techniques to spectral methods on other domains using multidimensional OPs. Further, we demonstrate how the Vector Spherical Harmonics can be used as an orthogonal basis for vector valued functions lying in the tangent space of the sphere, and thus how one can additionally derive gradient and divergence operators.
%
%We move on to working in 2D, where in recent years sparse spectral methods for solving PDEs have been derived using hierarchies of classical orthogonal polynomials on intervals, disks, and triangles. Presenting a new framework for choosing a suitable orthogonal polynomial basis for more general 2D domains defined via an algebraic curve as a boundary, this work builds on the observation that sparsity is guaranteed due to this definition of the boundary, and that the entries of partial differential operators can be determined using formulae in terms of (non-classical) univariate orthogonal polynomials. Triangles and the full disk are then special cases of our framework, which we formalise for the disk-slice and trapezium cases.
%
%Finally, we return the the surface of the sphere, namely the subdomain we call the spherical cap. Piecing together techniques and OPs from both the Spherical Harmonics and disk-slice work, we once again present a new orthogonal polynomial basis and sparse spectral method for the spherical cap, complete with the same observation about the guaranteed sparsity of differential and other operators. The motivation is for one to use spherical caps and spherical bands (which are a simple extension) as elements in a spectral element method for the sphere, with many applications in meteorology and astrophysics -- in particular, as a potential replacement of the Spherical Harmonics approach that is currently in use at the European Centre for Medium-range Weather Forecasts (ECMWF) which is predicted in the future to be too costly in the due to a parallel scalability bottleneck arising from the global spectral transform.

This thesis develops sparse spectral methods for solving partial differential equations (PDEs) on various multidimensional domains, with a specific focus on the disk-slice and trapezium in 2D, and the spherical cap in 3D. For the latter, the PDEs are surface PDEs involving Laplace-Beltrami operators, spherical gradients and other spherical operators.

We begin with an introduction to sparse spectral methods via viewing spherical harmonics as multidimensional orthogonal polynomials in $x$, $y$, $z$. We explain how differential operators can be applied as banded-block-banded matrix operators to coefficient vectors for a function's expansion. Further, we demonstrate how vector spherical harmonics in $x$, $y$ and $z$ can be used as an orthogonal basis for vector-valued functions, yielding similar banded-block-banded gradient and divergence operators.

We move on to presenting a new framework for choosing a suitable orthogonal polynomial basis for more general 2D domains defined via an algebraic curve as a boundary. This work builds on the observation that sparsity is guaranteed due to this definition of the boundary, and that the entries of partial differential operators can be determined using formulae in terms of (non-classical) univariate orthogonal polynomials. Triangles and the full disk are then special cases of our framework, which we formalise for the disk-slice and trapezium.

Piecing together the techniques used thus far, we present a new orthogonal polynomial basis and sparse spectral method for the spherical cap, complete with the same observation of the guaranteed sparsity of operators. The motivation is for one to use spherical caps bands as in a spectral element method for the sphere, with many applications in meteorology and astrophysics -- in particular, as a potential replacement of the spherical harmonics approach currently in use at ECMWF (which is predicted to become too costly due to a parallel scalability bottleneck arising from the global spectral transform).

\end{abstract}
