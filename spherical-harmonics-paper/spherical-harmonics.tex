\documentclass[11pt, oneside]{article}   	% use "amsart" instead of "article" for AMSLaTeX format
\usepackage{geometry}                		% See geometry.pdf to learn the layout options. There are lots.
\geometry{letterpaper}                   		% ... or a4paper or a5paper or ... 
%\geometry{landscape}                		% Activate for rotated page geometry
\usepackage[parfill]{parskip}    		% Activate to begin paragraphs with an empty line rather than an indent
\usepackage{graphicx}				% Use pdf, png, jpg, or eps§ with pdflatex; use eps in DVI mode
								% TeX will automatically convert eps --> pdf in pdflatex		
\usepackage{caption}
\usepackage{subcaption}
\usepackage{float}
\usepackage{amssymb}
\usepackage{amsmath}
\usepackage{bm}
\usepackage{bbm}
\usepackage{mleftright}

%SetFonts

%SetFonts

\usepackage{natbib}
\usepackage{url}
\bibliographystyle{elsarticle-harv}

\newcommand{\half}{\frac{1}{2}}
\newcommand{\R}{\mathbb{R}}
\newcommand{\C}{\mathbb{C}}
\newcommand{\Z}{\mathbb{Z}}
\newcommand{\N}{\mathbb{N}}
\newcommand{\No}{\mathbb{N}_0}
\newcommand{\Ylm}{Y^m_l}
\newcommand{\Ylmfull}{Y^m_l(\theta,\varphi)}
\newcommand{\Plm}{P^m_l}
\newcommand{\costheta}{\cos\theta}
\newcommand{\sintheta}{\sin\theta}
\newcommand{\cosphi}{\cos\varphi}
\newcommand{\sinphi}{\sin\varphi}
\newcommand{\eimphi}{e^{im\varphi}}
\newcommand{\alphalm}{\alpha^m_l}
\newcommand{\clm}{c^m_l}
\newcommand{\ctilde}{\tilde{c}^m_l}
\newcommand{\ctildemod}{\tilde{c}^{|m|}_l}
\newcommand{\chat}{\hat{c}^m_l}
\newcommand{\chatmod}{\hat{c}^{|m|}_l}
\newcommand{\ddx}{\frac{\mathrm{d}}{\mathrm{d}x}}
\newcommand{\dmdxm}{\frac{\mathrm{d}^m}{\mathrm{d}x^m}}

\newcommand{\Atilde}{\tilde{A}_{l,m}}
\newcommand{\Btilde}{\tilde{B}_{l,m}}
\newcommand{\Dtilde}{\tilde{D}_{l,m}}
\newcommand{\Etilde}{\tilde{E}_{l,m}}
\newcommand{\Ftilde}{\tilde{F}_{l,m}}
\newcommand{\Gtilde}{\tilde{G}_{l,m}}
\newcommand{\Alm}{A_{l,m}}
\newcommand{\Blm}{B_{l,m}}
\newcommand{\Dlm}{D_{l,m}}
\newcommand{\Elm}{E_{l,m}}
\newcommand{\Flm}{F_{l,m}}
\newcommand{\Glm}{G_{l,m}}

\newcommand{\xione}{\xi^{(1)}_{n, \lambda}}
\newcommand{\xitwo}{\xi^{(2)}_{n, \lambda}}
\newcommand{\xithree}{\xi^{(3)}_{n, \lambda}}
\newcommand{\xifour}{\xi^{(4)}_{n, \lambda}}

\newcommand{\bigP}{\mathbb{P}}
\newcommand{\Pl}{\mathbb{P}_l}
\newcommand{\gradP}{\nabla\mathbb{P}}
\newcommand{\gradPl}{\nabla\mathbb{P}_l}
\newcommand{\gradY}{\nabla Y}
\newcommand{\gradYlm}{\nabla Y^m_l}
\newcommand{\gradpY}{\nabla^\perp Y}
\newcommand{\gradpYlm}{\nabla^\perp Y^m_l}

\newcommand{\Dlt}{D^T_l}

\newcommand{\curlyy}{\bm{\mathcal{Y}}}
\newcommand{\blone}{\beta_{l, 1}}
\newcommand{\blzero}{\beta_{l, 0}}
\newcommand{\blmone}{\beta_{l, -1}}
\newcommand{\chivec}{\bm{\chi}_{1,m_s}}
\newcommand{\cgcoeff}{\mathcal{C}}

\newcommand{\alm}{a_{l,m}}
\newcommand{\blm}{b_{l,m}}
\newcommand{\dlm}{d_{l,m}}
\newcommand{\elm}{e_{l,m}}
\newcommand{\flm}{f_{l,m}}
\newcommand{\glm}{g_{l,m}}
\newcommand{\hlm}{h_{l,m}}
\newcommand{\jlm}{j_{l,m}}
\newcommand{\klm}{k_{l,m}}
\newcommand{\almperp}{a_{l,m}^\perp}
\newcommand{\blmperp}{b_{l,m}^\perp}
\newcommand{\dlmperp}{d_{l,m}^\perp}
\newcommand{\elmperp}{e_{l,m}^\perp}
\newcommand{\flmperp}{f_{l,m}^\perp}
\newcommand{\glmperp}{g_{l,m}^\perp}
\newcommand{\hlmperp}{h_{l,m}^\perp}
\newcommand{\jlmperp}{j_{l,m}^\perp}
\newcommand{\klmperp}{k_{l,m}^\perp}

\newcommand{\unitvec}{\hat{\bm{k}}}





\title{Spherical Harmonics on the Unit Sphere}
\author{Ben Snowball}
%\date{}							% Activate to display a given date or no date


\begin{document}

\maketitle



\section{Introduction}

\subsection{On the Unit Interval (1D)}

On the unit interval, \([0,1]\), we note that there is a hierarchy of orthogonal polynomials (OPs) such that:
\begin{align}
& \ddx P^{(a,b)}_l (x) = \text{const.} \times P^{(a+1,b+1)}_{l-1}(x) \\
\implies & \dmdxm P_l(x) = \text{const.} \times P^{(m,m)}_{l-m}(x)
\end{align}
where \(P^{(a,b)}_l (x)\) is the \(l\) degree Jacobi polynomial, orthogonal with weight \(w(x) = (1-x)^a(1+x)^b\), and \(P_l(x) := P^{(0,0)}_l (x)\) is the Legendre polynomial of degree \(l\).

Further, we can define associated Legendre polynomials that are also orthogonal:
\begin{align}
\Plm(x) &:= (-1)^m (1-x^2)^\frac{m}{2} \dmdxm P_l(x) = \chat (1-x^2)^\frac{m}{2} P^{(m,m)}_{l-m}(x) \\
P^{-m}_l(x) &:= \ctilde \Plm(x),
\end{align}
for \( m \in \No\) where
\begin{align}
\chat &:= \frac{\Gamma(l+m+1)}{(-2)^m\Gamma(l+1)} \\
\ctilde &:= \frac{(-1)^m\Gamma(l-m+1)}{\Gamma(l+m+1)}
\end{align}
using the gamma function \(\Gamma(n) := (n-1)!\) for \(n \in \N\).

\subsection{On the Unit Sphere (3D)}

Let \((x,y,z)\) be a point on the unit sphere such that \(x^2+y^2+z^2 = 1\). Let \(\theta, \varphi\) denote the angles such that
\begin{align}
x &= \sintheta \cosphi \\
y &= \sintheta \sinphi \\
z &= \costheta.
\end{align}
Further, define
\begin{align}
\clm := \Bigg(\frac{(2l+1)\Gamma(l-m+1)}{4\pi\Gamma(l+m+1)}\Bigg)^\frac{1}{2},
\end{align}
and
\begin{align}
\alphalm := \begin{cases} 
		\clm \chatmod \quad \quad \quad \text{if } m \ge 0 \\
		\clm \chatmod \ctildemod \quad \text{if } m < 0
	   \end{cases}.
\end{align}

We can the define the spherical harmonics, orthogonal on the unit sphere as:
\begin{align}
\Ylmfull &:= \clm \eimphi \Plm (\costheta) \\
&= \alphalm (1 - (\costheta)^2)^\frac{|m|}{2} \eimphi P^{(|m|,|m|)}_{l-|m|}(\costheta), \quad \text{where } 0 \le |m| \le l, \, l \in \No.
\end{align} 
Note that we can then express \(\Ylm\) in terms of \(x,y,z\) instead of \(\theta, \varphi\) by noting that \(\costheta = z\) and that \( \eimphi\) can be expressed in terms of \(x,y,z\) for any \(m\in\mathbb{Z}\).



\section{Surface of the sphere}

\subsection{Deriving expressions for the multiplication by \(x, y, z\) of the spherical harmonics}

We start by establishing some equations to be used to find \(x\,\Ylm(x,y,z)\) etc. in terms of \(Y^{m'}_{l'}(x,y,z)\) for some point \((x,y,z)\) on the unit circle.

Using (7)-(9), we can write
\begin{align}
\Ylm(x,y,z) &= \alphalm (1 - z^2)^\frac{|m|}{2} \eimphi P^{(|m|,|m|)}_{l-|m|}(z), \quad \text{where } 0 \le |m| \le l, \, l \in \No.
\end{align}

Note that the recurrence relationship for the Jacobi polynomials satisfies
\begin{align}
z P^{(m,m)}_k(z) = \frac{1}{\kappa_{k,m}} \big[ P^{(m,m)}_{k+1}(z) - \lambda_{k,m} P^{(m,m)}_k(z) + \mu_{k,m} P^{(m,m)}_{k-1}(z) \big],
\end{align}
for \(k\ge0, \, m\in\Z\), where
\begin{align}
\kappa_{k,m} &:= \frac{(2k+2m+1)(k+m+1)}{(k+1)(k+2m+1)}, \\
\lambda_{k,m} &:= 0\\
\mu_{k,m} &:= \frac{k+m+1}{(k+1)(k+2m+1)}.
\end{align}
Thus,
\begin{align}
z P^{(m,m)}_{l-m}(z) = \Ftilde P^{(m,m)}_{l-m+1}(z) + \Gtilde P^{(m,m)}_{l-m-1}(z),
\end{align}
for \(k\ge0, \, m\in\Z\), where
\begin{align}
\Ftilde &:= \frac{(l-m+1)(l+m+1)}{(2l+1)(l+1)}, \\
\Gtilde &:= \begin{cases}
			\frac{l}{2l+1} \quad \text{if } l - m \ge 1 \\
			\quad 0 \quad \quad \text{else}
		\end{cases} .
\end{align}

Further, note that
\begin{align}
\cosphi \, \eimphi &= \frac{1}{2} (e^{i\varphi} + e^{-i\varphi}) \eimphi =  \frac{1}{2} (e^{i(m+1)\varphi} + e^{i(m-1)\varphi}) \\
\sinphi \, \eimphi &= \frac{1}{2i} (e^{i\varphi} - e^{-i\varphi}) \eimphi =  \frac{-i}{2} (e^{i(m+1)\varphi} - e^{i(m-1)\varphi})
\end{align}

Finally, note the relationship between the Jacobi polynomials \(P^{(a,b)}_{l}(z)\) and the Ultraspherical polynomials \(C^{(\lambda)}_{n}(z)\), as well as the three-term recurrence relation for the Ultraspherical polynomials, i.e. for \(m \in \No\):
\begin{align}
\nu_{l,m} C^{(m+1/2)}_{l-m}(z) = P^{(m,m)}_{l-m}(z),
\end{align}
and
\begin{align}
C^{(\lambda)}_n(z) &= \xi^{(1)}_{n, \lambda} \big[ C^{(\lambda+1)}_n(z) - C^{(\lambda+1)}_{n-2}(z) \big]  \\
(1-z^2) \, C^{(\lambda)}_n(z) &= \frac{1}{\xi^{(2)}_{n, \lambda}} \big[ \xi^{(3)}_{n, \lambda} C^{(\lambda-1)}_n(z) - \xi^{(4)}_{n, \lambda}C^{(\lambda-1)}_{n+2}(z) \big] ,
\end{align}
where
\begin{align}
\nu_{l,m} &:= \frac{\Gamma(l+1)\Gamma(2m+1)}{\Gamma(l+m+1)\Gamma(m+1)} \\
\xione&:= \frac{\lambda}{n+\lambda} \\
\xitwo &:= 4(\lambda-1)(n+\lambda) \\
\xithree &:= (n+2\lambda-2)(n+2\lambda-1) \\
\xifour &:= (n+1)(n+2).
\end{align}

Thus, we can write three-term recurrences for the Jacobi polynomials as:
\begin{align}
P^{(m,m)}_{l-m}(z) &= \Atilde P^{(m+1,m+1)}_{l-m}(z) + \Btilde P^{(m+1,m+1)}_{l-m-2}(z) \\
(1-z^2) \, P^{(m,m)}_{l-m}(z) &= \Dtilde P^{(m-1,m-1)}_{l-m+2}(z) + \Etilde P^{(m-1,m-1)}_{l-m}(z),
\end{align}
for \(l, m \in \No\) s.t. \(0 \le m \le l\), where
\begin{align}
\Atilde &:= \frac{(l+m+2)(l+m+1)}{2(2l+1)(l+1)} \\
\Btilde &:= \begin{cases}
			- \frac{l}{2(2l+1)} \quad \text{if } l - m \ge 2 \\
			\quad \quad 0 \quad \quad \text{else}
		\end{cases} \\
\Dtilde &:= - \frac{2(l-m+2)(l-m+1)}{(2l+1)(l+1)} \\
\Etilde &:= \frac{2l}{2l+1} .
\end{align}

We can now write down expressions for the multiplication of \(\Ylm(x,y,z)\) by \(x, y \) or \(z\) for some point \((x,y,z)\) on the unit sphere for \(l \in \No\), \(m \in \Z\) s.t. \(0 \le |m| \le l\) as follows:
\begin{align}
x\,\Ylm(x,y,z) &= \alphalm \cosphi \eimphi \sintheta (1-z^2)^{|m|/2} P^{(|m|,|m|)}_{l-|m|}(z) \nonumber \\
		     &= \half \alphalm (e^{i(m+1)\varphi} + e^{i(m-1)\varphi}) (1-z^2)^{\frac{|m|+1}{2}} P^{(|m|,|m|)}_{l-|m|}(z) \nonumber \\
		     &= \half \alphalm e^{i(m+1)\varphi} (1-z^2)^{\frac{|m|+1}{2}} \big[ \Atilde P^{(|m|+1,|m|+1)}_{l-|m|}(z) + \Btilde P^{(|m|+1,|m|+1)}_{l-|m|-2}(z) \big] \nonumber \\
		     &\quad \quad +  \half \alphalm e^{i(m-1)\varphi} (1-z^2)^{\frac{|m|-1}{2}} \big[ \Dtilde P^{(|m|-1,|m|-1)}_{l-|m|+2}(z) + \Etilde P^{(|m|-1,|m|-1)}_{l-|m|}(z) \big] \nonumber \\
		     &= \Alm Y^{m+1}_{l+1}(x,y,z) +  \Blm Y^{m+1}_{l-1}(x,y,z) \nonumber \\
		     & \quad \quad \quad + \Dlm Y^{m-1}_{l+1}(x,y,z) + \Elm Y^{m-1}_{l-1}(x,y,z),
\end{align}
\begin{align}
y\,\Ylm(x,y,z) &= \alphalm \sinphi \eimphi \sintheta (1-z^2)^{|m|/2} P^{(|m|,|m|)}_{l-|m|}(z) \nonumber \\
		     &= -\half i \alphalm (e^{i(m+1)\varphi} - e^{i(m-1)\varphi}) (1-z^2)^{\frac{|m|+1}{2}} P^{(|m|,|m|)}_{l-|m|}(z) \nonumber \\
		     &= - i \, \big[\Alm Y^{m+1}_{l+1}(x,y,z) +  \Blm Y^{m+1}_{l-1}(x,y,z) \big] \nonumber \\
		     &\quad \quad \quad + i \, \big[ \Dlm Y^{m-1}_{l+1}(x,y,z) + \Elm Y^{m-1}_{l-1}(x,y,z) \big],
\end{align}
\begin{align}
z\,\Ylm(x,y,z) &= \alphalm \eimphi (1-z^2)^{|m|/2} z P^{(|m|,|m|)}_{l-|m|}(z) \nonumber \\
		     &= \alphalm \eimphi (1-z^2)^{|m|/2} \big[ \Ftilde P^{(|m|,|m|)}_{l-|m|+1}(z) + \Gtilde P^{(|m|,|m|)}_{l-|m|-1}(z) \big]\nonumber \\
		     &= \Flm Y^{m}_{l+1}(x,y,z) + \Glm Y^{m}_{l-1}(x,y,z) ,
\end{align}
where
\begin{align}
\Alm &:= \begin{cases}
			\frac{\alphalm}{2\alpha^{m+1}_{l+1}} \Atilde \quad \text{if } m\ge0 \\
			\frac{\alphalm}{2\alpha^{m+1}_{l+1}} \tilde{D}_{l,|m|} \quad \text{if } m<0 
	       \end{cases} \\
\Blm &:= \begin{cases}
			\frac{\alphalm}{2\alpha^{m+1}_{l-1}} \Btilde \quad \text{if } m\ge0, \, l - |m| \ge 2 \\
			\frac{\alphalm}{2\alpha^{m+1}_{l-1}} \tilde{E}_{l,|m|} \quad \text{if } m<0 \\
			\quad \quad 0 \quad \quad \quad \quad \text{else} 
	        \end{cases} \\
\Dlm &:= \begin{cases}
			\frac{\alphalm}{2\alpha^{m-1}_{l+1}} \Dtilde \quad \text{if } m>0 \\
			\frac{\alphalm}{2\alpha^{m-1}_{l+1}} \tilde{A}_{l,|m|} \quad \text{if } m\le0 
	       \end{cases} \\
\Elm &:= \begin{cases}
			\frac{\alphalm}{2\alpha^{m-1}_{l-1}} \Etilde \quad \text{if } m>0 \\
			\frac{\alphalm}{2\alpha^{m-1}_{l-1}} \tilde{B}_{l,|m|} \quad \text{if } m\le0 , \, l - |m| \ge 2 \\
			\quad \quad 0 \quad \quad \quad \quad \text{else} 
	        \end{cases} \\ 
\Flm &:= \frac{\alphalm}{\alpha^{m}_{l+1}} \Ftilde \\
\Glm &:= \frac{\alphalm}{\alpha^{m}_{l-1}} \Gtilde.
\end{align}



\subsection{Jacobi matrices}

Define, for \(l \in \No\), \(\Pl\) as the column vector of the degree \(l\) spherical harmonic polynomials, and \(\bigP\) as the stacked block vector of the \(\Pl\)'s; that is
\begin{align}
\Pl := \begin{bmatrix}
		Y^{-l}_l \\
		\vdots \\
		Y^l_l
	\end{bmatrix} \in \C^{2l+1}, 
\quad \quad 
\bigP := \begin{bmatrix}
		\bigP_0 \\
		\hline
		\bigP_1 \\
		\hline
		\bigP_2 \\
		\vdots \\
	\end{bmatrix}.
\end{align}

Define the (Jacobi) matrices \(J^x, J^y, J^z\) by 
\begin{align}
J^x \bigP = x \bigP, \quad J^y \bigP = y \bigP, \quad J^z \bigP = z \bigP.
\end{align}
Then, using equations (38-40), we have that the Jacobi matrices take the following block-tridiagonal form:
\begin{align}
J^x &= \begin{bmatrix}
		B^x_0 & A^x_0 & & & & \\
		C^x_1 & B^x_1 & A^x_1 & & & \\
		& C^x_2 & B^x_2 & A^x_2  & & & \\
		& & C^x_3 & \ddots & \ddots & \\
		& & & \ddots & \ddots & \ddots \\
	\end{bmatrix}, \\
J^y &= \begin{bmatrix}
		B^y_0 & A^y_0 & & & & \\
		C^y_1 & B^y_1 & A^y_1 & & & \\
		& C^y_2 & B^y_2 & A^y_2  & & & \\
		& & C^y_3 & \ddots & \ddots & \\
		& & & \ddots & \ddots & \ddots \\
	\end{bmatrix}, \\
J^z &= \begin{bmatrix}
		B^z_0 & A^z_0 & & & & \\
		C^z_1 & B^z_1 & A^z_1 & & & \\
		& C^z_2 & B^z_2 & A^z_2  & & & \\
		& & C^z_3 & \ddots & \ddots & \\
		& & & \ddots & \ddots & \ddots \\
	\end{bmatrix},
\end{align}
where for \(l \in \No\),
\begin{align}
A^x_l &:= \begin{bmatrix}
		D_{l,-l} & 0 & A_{l,-l} & & \\
		& \ddots & \ddots & \ddots & \\
		& & D_{l,l} & 0 & A_{l,l} \\
	    \end{bmatrix} \in \R^{(2l+1)\times(2l+3)}, \\
B^x_l &:= 0 \in \R^{(2l+1)\times(2l+1)} \\
C^x_l &:= \begin{bmatrix}
		B_{l,-l} & & \\
		0 & \ddots & \\
		E_{l,-l+2} & \ddots & B_{l,l-2} \\
		& \ddots & 0 \\
		& & E_{l,l}
	    \end{bmatrix} \in \R^{(2l+1)\times(2l-1)} \quad (l \ne 0), \\
A^y_l &:= -i \begin{bmatrix}
		-D_{l,-l} & 0 & A_{l,-l} & & \\
		& \ddots & \ddots & \ddots & \\
		& & -D_{l,l} & 0 & A_{l,l} \\
	    \end{bmatrix} \in \C^{(2l+1)\times(2l+3)}, \\
B^y_l &:= 0 \in \R^{(2l+1)\times(2l+1)} \\
C^y_l &:= -i \begin{bmatrix}
		B_{l,-l} & & \\
		0 & \ddots & \\
		-E_{l,-l+2} & \ddots & B_{l,l-2} \\
		& \ddots & 0 \\
		& & -E_{l,l}
	    \end{bmatrix} \in \C^{(2l+1)\times(2l-1)} \quad (l \ne 0), \\
A^z_l &:= \begin{bmatrix}
		0 & F_{l,-l} & 0 & & \\
		& \ddots & \ddots & \ddots & \\
		& & 0 & F_{l,l} & 0 \\
	    \end{bmatrix} \in \R^{(2l+1)\times(2l+3)}, \\
B^z_l &:= 0 \in \R^{(2l+1)\times(2l+1)} \\
C^z_l &:= \begin{bmatrix}
		0 & & \\
		G_{l,-l+1} & \ddots & \\
		0 & \ddots & 0 \\
		& \ddots & G_{l,l-1} \\
		& & 0
	    \end{bmatrix} \in \R^{(2l+1)\times(2l-1)} \quad (l \ne 0). \\
\end{align}




\subsection{Three-term recurrence relation for \(\bigP\)}

Combining each system in (48) we can write the tridiagonal-block system
\begin{align}
\renewcommand\arraystretch{1.3}
\mleft[
\begin{array}{c|c|cc}
		1  & & & \\
		\hline 
		B^x_0 - xI_1 & A^x_0 & & \\
		B^y_0 - yI_1 & A^y_0 & & \\
		B^z_0 - zI_1 & A^z_0 & & \\
		\hline 
		C^x_1 & B^x_1 - xI_{3} & \quad A^x_1 \quad & \\
		C^y_1 & B^y_1 - yI_{3} & \quad A^y_1 \quad & \\
		C^z_1 & B^z_1 - zI_{3} & \quad A^z_1 \quad & \\
		\hline
		& C^x_2 & \ddots & \ddots \\
		& C^y_2 & & \\
		& C^z_2 & & \\
		\hline
		& & \ddots &
\end{array}
\mright]
\bigP
=
\begin{bmatrix}
	\alpha^0_0 \\ 0 \\ 0 \\ 0 \\ \\ \vdots \\ \\ \vdots \\ \\ \vdots \\ \\
\end{bmatrix},
\end{align}
where \(I_{2l+1}\) is the \((2l+1)\times(2l+1)\) identity matrix.

Let us define the joint matrices that comprise each block. For each \(l \in \No\):
\begin{align}
A_l &:= \begin{bmatrix}
		A^x_l \\
		A^y_l \\
		A^z_l
	    \end{bmatrix} \in \R^{3(2l+1)\times(2l+3)}, \quad
C_l := \begin{bmatrix}
		C^x_l \\
		C^y_l \\
		C^z_l
	    \end{bmatrix} \in \R^{3(2l+1)\times(2l-1)} \quad (l \ne 0), \\
B_l &:= \begin{bmatrix}
		B^x_l \\
		B^y_l \\
		B^z_l
	    \end{bmatrix} \in \R^{3(2l+1)\times(2l+1)}, \quad
G_l := \begin{bmatrix}
		xI_{2l+1} \\
		yI_{2l+1} \\
		zI_{2l+1}
	    \end{bmatrix} \in \R^{3(2l+1)\times(2l+1)}.
\end{align}
Then our system (62) simply becomes
\begin{align}
\renewcommand\arraystretch{1.3}
\mleft[
\begin{array}{cccc}
		1  & & & \\
		B_0-G_0 & A_0 & & \\
		C_1 & B_1-G_1 & \quad A_1 \quad & \\
		& C_2 & \ddots & \ddots \\
		& & \ddots &
\end{array}
\mright]
\bigP
=
\begin{bmatrix}
	\alpha^0_0 \\ 0 \\ 0 \\ \vdots \\ \\
\end{bmatrix}.
\end{align}

For each \(l \in \No\) let \(\Dlt\) be any matrix that is a left inverse of \(A_l\), i.e. such that \(\Dlt A_l = I_{2l+3}\). Multiplying our system by the preconditioner matrix that is given by the block diagonal matrix of the \(\Dlt\)'s, we obtain the lower triangular system
\begin{align}
\renewcommand\arraystretch{1.3}
\mleft[
\begin{array}{cccc}
		1  & & & \\
		D^T_0(B_0-G_0) & I_1 & & \\
		D^T_1 C_1 & D^T_1(B_1-G_1) & \quad I_3 \quad & \\
		& D^T_2 C_2 & \ddots & \ddots \\
		& & \ddots &
\end{array}
\mright]
\bigP
=
\begin{bmatrix}
	\alpha^0_0 \\ 0 \\ 0 \\ \vdots \\ \\
\end{bmatrix}.
\end{align}

Expanding this we can solve the system (66) to find a three-term recurrence relation for each \(\bigP_{l+1}\) in terms of the previous two sub-vectors \(\Pl\) and \(\bigP_{l-1}\):
\begin{align}
\begin{cases}
\bigP_{-1} := 0 \\
\bigP_{0} := \alpha^0_0 \\
\bigP_{l+1} = -\Dlt (B_l-G_l) \Pl - \Dlt C_l  \,\bigP_{l-1}, \quad l \in \N.
\end{cases}
\end{align}

We note that we can choose the matrices \(\Dlt\) in the following way. For \(l \in \N\), we set
\begin{align}
\Dlt = \begin{bmatrix}
		\hat{A}^{x,y}_l & 0_{(2l+3)\times(2l+1)}
	  \end{bmatrix} \in \R^{(2l+3)\times3(2l+1)}
\end{align}
where \(0_{(2l+3)\times(2l+1)}\) the zero matrix in \(\R^{(2l+3)\times(2l+1)}\), and \(\hat{A}^{x,y}_l \in \R^{(2l+3)\times2(2l+1)}\) is the left inverse of the matrix \(\begin{bmatrix} A^x_l \\ A^y_l \end{bmatrix}\), given by
\begin{align}
\hat{A}^{x,y}_l = \begin{bmatrix}
		\frac{1}{2D_{l,-l}} & 0 & \hdots & 0 & -\frac{i}{2D_{l,-l}} & 0 & \hdots & 0 \\
		& \ddots & & & & \ddots & & \\
		& & \ddots & & & & \ddots & \\
		& & & \frac{1}{2D_{l,l}} & 0 & \hdots & 0 & -\frac{i}{2D_{l,l}} \\
		0 & \hdots & \frac{1}{2A_{l,l-1}} & 0 & \hdots & 0 & \frac{i}{2A_{l,l-1}} & 0 \\
		0 & \hdots & 0 & \frac{1}{2A_{l,l}} & 0 & \hdots & 0 & \frac{i}{2A_{l,l}}
	  \end{bmatrix}.
\end{align}
For \(l = 0\) we set
\begin{align}
D^T_0 = \begin{bmatrix}
		\frac{1}{2D_{0,0}}&  -\frac{i}{2D_{0,0}} & 0 \\
		0 & 0 & \frac{1}{F_{0,0}} \\
		\frac{1}{2A_{0,0}}&  \frac{i}{2A_{0,0}} & 0 
	     \end{bmatrix}.
\end{align}




\subsection{Evaluation of a scalar function on the sphere}
We can use the Clenshaw algorithm to evaluate a function at a given point \((x,y,z)\) on the unit sphere provided we know the coefficients of the function when expanded in the spherical harmonic basis, i.e. suppose \(f(x,y,z)\) is a function and we know the set \(\{\bold{f}_l\}\) s.t.
\begin{align}
f(x,y,z) \approx \sum^N_{l=0} \bold{f}_l^T \, \Pl (x,y,z), \quad \text{where } \Pl (x,y,z), \bold{f}_l \in \R^{2l+1} \text{ foreach } l \in \{0,\dots,N\}.
\end{align}

The Clenshaw algorithm is then as follows:
\begin{align*}
\quad &\text{1) } \text{Set } \bm{\gamma}_{N+2} = \bold{0}, \: \bm{\gamma}_{N+2} = \bold{0}. \\
\quad &\text{2) } \text{For } n = N:-1:1 \\
\quad & \quad \quad \quad \text{set } \bm{\gamma}_{n}^T = \bold{f}_n^T - \bm{\gamma}_{n+1}^T D^T_n (B_n - G_n) -  \bm{\gamma}_{n+2}^T D^T_{n+1}C_{n+1} \\
\quad &\text{3) } \text{Output: } f(x,y,z) \approx \bigP_0 f_0 + \bm{\gamma}_{1}^T \bigP_1 - \bigP_0 \bm{\gamma}_{2}^T D^T_1 C_1.
\end{align*}






\section{Tangent space}

Since the spherical harmonics are a basis for the surface of the sphere, and the tangent space of the sphere is spanned by the gradient  and perpendicular gradient of a scalar function, we have that the gradients and perpendicular gradients of the spherical harmonics are a basis for the tangent space, namely \(\gradYlm\), \(\gradpYlm\}\). Note that the perpendicular gradient is related to the regular surface gradient by
\begin{align}
\gradpYlm(x,y,z) = \unitvec \times \gradYlm(x,y,z),
\end{align}
where \(\unitvec\) is the unit vector normal to the surface of the sphere at the point \((x,y,z)\), i.e. as we are looking at the unit sphere, \(\unitvec\) is simply given by
\begin{align}
\unitvec = \begin{bmatrix} x \\ y \\ z \end{bmatrix}.
\end{align}



\subsection{Spin-1 tensor spherical harmonics}

The gradient and perpendicular gradient of a spherical harmonic \(\Ylm(x,y,z)\) can be expressed in terms of spin-1 tensor spherical harmonics, which in turn can each be expressed as a vector-weighted sum of spherical harmonics.

We start by defining what we mean by a spin-1 tensor spherical harmonic. In general, the tensor spherical harmonic is given by, for \(2l, \, j, \, 2s \in \No\),
\begin{align}
\curlyy^{j,s}_{l,m}(x,y,z) = \sum_{m_s=-s}^s < j \quad m-m_s \; ; \; s \quad m_s \; | \; l \quad m > \, Y^{m-m_s}_j (x,y,z) \, \bm{\chi}_{s, m_s},
\end{align}
where \(\bm{\chi}_{s, m_s}\) are the simultaneous eigenstates of the spin operators \(\bold{S}^2\) and \(S_z\), and where \(< j \quad m-m_s \; ; \; 1 \quad m_s \; | \; l \quad m >\) is a Clebsch-Gordan coefficient. We note that a property of the Clebsch-Gordan coefficients means that they vanish unless \(|j - s| \le l \le j + s\). We further note that there are simple calculable expressions for the Clebsch-Gordan coefficients when the spin \(s=1\).

Thus we have that the three spin-1 tensor spherical harmonics are given by
\begin{align}
\curlyy^{l \pm1,s}_{l,m}(x,y,z) &= \sum_{m_s=-1}^1 < l \pm 1 \quad m-m_s \; ; \; 1 \quad m_s \; | \; l \quad m > \, Y^{m-m_s}_{l \pm 1} (x,y,z) \, \chivec, \\
\curlyy^{l,s}_{l,m}(x,y,z) &= \sum_{m_s=-1}^1 < l \quad m-m_s \; ; \; 1 \quad m_s \; | \; l \quad m > \, Y^{m-m_s}_l (x,y,z) \, \chivec.
\end{align}

Here, the vectors \(\chivec\) are the orthonormal eigenvectors of the spin-1spin matrix 
\begin{align}
S_3 = \begin{bmatrix} 0 & -i & 0 \\ i & 0 & 0 \\ 0 & 0 & 0 \end{bmatrix},
\end{align}
and so are given as
\begin{align}
\bm{\chi}_{1, \pm 1} = \frac{1}{\sqrt{2}} \begin{bmatrix} \mp 1 \\ -i \\ 0 \end{bmatrix}, \quad \bm{\chi}_{1,0} = \begin{bmatrix} 0 \\ 0 \\ 1 \end{bmatrix}.
\end{align}

Then, for any \(l \in \No\), \(m \in \Z\) s.t. \(0 \le |m| \le l\) we have that
\begin{align}
\nabla \Ylm &= \blmone \, \curlyy^{l-1,1}_{l,m} + \blone \curlyy^{l+1,1}_{l,m}, \\
\nabla^\perp \Ylm &= \blzero \, \curlyy^{l,1}_{l,m},
\end{align}
where
\begin{align}
\blmone &:= (l+1) \, \Big(\frac{l}{2l+1}\Big)^\half, \quad
\blzero := i \, \big(l(l+1)\big)^\half, \quad
\blone := l \, \Big(\frac{l+1}{2l+1}\Big)^\half.
\end{align}


\subsection{Deriving expressions for the multiplication by \(x, y, z\) of \(\nabla \Ylm\), \(\nabla^\perp \Ylm\) }

We start by establishing some equations to be used to find \(x \,\gradYlm (x,y,z)\), \(x \,\gradpYlm (x,y,z)\) etc. in terms of \(\gradY^{m'}_{l'} (x,y,z)\), \(\gradpY^{m'}_{l'} (x,y,z)\). First, for compactness of notation, we define
\begin{align}
\cgcoeff^{L, m_s}_{l, m} := \, < L \quad m-m_s \; ; \; 1 \quad m_s \; | \; l \quad m >.
\end{align}

Now, we have that
\begin{align}
x \,\gradYlm 
&= x \, \blmone \, \curlyy^{l-1,1}_{l,m} + x \, \blone \curlyy^{l+1,1}_{l,m} \nonumber
\\
&= \sum_{m_s=-1}^{1} \chivec \Big[ \blmone \, \cgcoeff^{l-1,m_s}_{l,m} \big\{ A_{l-1,m-m_s} \, Y^{m-m_s+1}_{l} + B_{l-1,m-m_s} \, Y^{m-m_s+1}_{l-2} \nonumber \\ 
& \quad \quad \quad \quad \quad \quad \quad \quad \quad \quad \quad \quad \quad \quad + D_{l-1,m-m_s} \, Y^{m-m_s-1}_{l} + E_{l-1,m-m_s} \, Y^{m-m_s-1}_{l-2} \big\} \nonumber \\
& \quad \quad \quad \quad \quad \quad \quad + \blone \, \cgcoeff^{l+1,m_s}_{l,m} \big\{ A_{l+1,m-m_s} \, Y^{m-m_s+1}_{l+2} + B_{l+1,m-m_s} \, Y^{m-m_s+1}_{l} \nonumber \\ 
& \quad \quad \quad \quad \quad \quad \quad \quad \quad \quad \quad \quad \quad \quad + D_{l+1,m-m_s} \, Y^{m-m_s-1}_{l+2} + E_{l+1,m-m_s} \, Y^{m-m_s-1}_{l} \big\} \Big]
\end{align}
\begin{align}
\implies x \,\gradYlm &= \alm \gradY^{m+1}_{l+1} + \blm \gradY^{m+1}_{l-1} + \dlm \gradY^{m-1}_{l+1} + \elm \gradY^{m-1}_{l-1} \nonumber \\
& \quad \quad \quad + \hlm \gradpY^{m+1}_{l} + \jlm \gradpY^{m-1}_{l},
\end{align}
where, for any valid \(m_s\) value,
\begin{align}
\alm &:= \frac{\blone}{\beta_{l+1,1}} \frac{\cgcoeff^{l+1,m_s}_{l,m}}{\cgcoeff^{l+2,m_s}_{l+1,m+1}} A_{l+1,m-m_s}, \\
\blm &:= \frac{\blmone}{\beta_{l-1,-1}} \frac{\cgcoeff^{l-1,m_s}_{l,m}}{\cgcoeff^{l-2,m_s}_{l-1,m+1}} B_{l-1,m-m_s}, \\
\dlm &:= \frac{\blone}{\beta_{l+1,1}} \frac{\cgcoeff^{l+1,m_s}_{l,m}}{\cgcoeff^{l+2,m_s}_{l+1,m-1}} D_{l+1,m-m_s}, \\
\elm &:= \frac{\blmone}{\beta_{l-1,-1}} \frac{\cgcoeff^{l-1,m_s}_{l,m}}{\cgcoeff^{l-2,m_s}_{l-1,m-1}} E_{l-1,m-m_s}, \\
\hlm &:= \frac{1}{\blzero \, \cgcoeff^{l,m_s}_{l, m+1}} \, \Big[ \cgcoeff^{l-1,m_s}_{l,m} \, \blmone \, A_{l-1,m-m_s} + \cgcoeff^{l+1,m_s}_{l,m} \, \blone \, B_{l+1,m-m_s} \nonumber \\
& \quad \quad \quad \quad \quad \quad \quad - \alm \, \beta_{l+1,-1} \, \cgcoeff^{l,m_s}_{l+1,m+1} - \blm \, \beta_{l-1,1} \, \cgcoeff^{l,m_s}_{l-1,m+1} \Big], \\
\jlm &:= \frac{1}{\blzero \, \cgcoeff^{l,m_s}_{l, m-1}} \, \Big[ \cgcoeff^{l-1,m_s}_{l,m} \, \blmone \, D_{l-1,m-m_s} + \cgcoeff^{l+1,m_s}_{l,m} \, \blone \, E_{l+1,m-m_s} \nonumber \\
& \quad \quad \quad \quad \quad \quad \quad - \dlm \, \beta_{l+1,-1} \, \cgcoeff^{l,m_s}_{l+1,m-1} - \elm \, \beta_{l-1,1} \, \cgcoeff^{l,m_s}_{l-1,m-1} \Big].
\end{align}

Note that (it can be shown) these are constants for each \(l,m\) pair despite appearing to depend on the value of \(m_s\); we need only use any valid \(m_s\) value for each expression. By "valid", we mean the Clebsch-Gordan coefficients do not vanish for that \(m_s\) value when used. 

Similarly, we have that
\begin{align}
y \,\gradYlm &= i \, \Big[ -\alm \gradY^{m+1}_{l+1} - \blm \gradY^{m+1}_{l-1} + \dlm \gradY^{m-1}_{l+1} + \elm \gradY^{m-1}_{l-1} \nonumber \\
& \quad \quad \quad - \hlm \gradpY^{m+1}_{l} + \jlm \gradpY^{m-1}_{l} \Big].
\end{align}

Further, we also have that
\begin{align}
z \,\gradYlm 
&= z \, \blzero \, \curlyy^{l,1}_{l,m} \nonumber
\\
&= \sum_{m_s=-1}^{1} \chivec \Big[ \blmone \, \cgcoeff^{l-1,m_s}_{l,m} \big\{ F_{l-1,m-m_s} \, Y^{m-m_s}_{l} + G_{l-1,m-m_s} \, Y^{m-m_s}_{l-2} \big\} \nonumber \\ 
& \quad \quad \quad \quad \quad \quad \quad + \blone \, \cgcoeff^{l+1,m_s}_{l,m} \big\{ F_{l+1,m-m_s} \, Y^{m-m_s}_{l+2} + G_{l+1,m-m_s} \, Y^{m-m_s}_{l} \big\} \Big]
\end{align}
\begin{align}
\implies z \,\gradYlm &= \flm \gradY^{m}_{l+1} + \glm \gradY^{m}_{l-1} + \klm \gradpY^{m}_{l},
\end{align}
where, for any valid \(m_s\) value,
\begin{align}
\flm &:= \frac{\blone}{\beta_{l+1,1}} \frac{\cgcoeff^{l+1,m_s}_{l,m}}{\cgcoeff^{l+2,m_s}_{l+1,m+1}} F_{l+1,m-m_s}, \\
\glm &:= \frac{\blmone}{\beta_{l-1,-1}} \frac{\cgcoeff^{l-1,m_s}_{l,m}}{\cgcoeff^{l-2,m_s}_{l-1,m+1}} G_{l-1,m-m_s}, \\
\klm &:= \frac{1}{\blzero \, \cgcoeff^{l,m_s}_{l, m}} \, \Big[ \cgcoeff^{l-1,m_s}_{l,m} \, \blmone \, F_{l-1,m-m_s} + \cgcoeff^{l+1,m_s}_{l,m} \, \blone \, G_{l+1,m-m_s}  \nonumber \\
& \quad \quad \quad \quad \quad \quad \quad - \flm \, \beta_{l+1,-1} \, \cgcoeff^{l,m_s}_{l+1,m} - \glm \, \beta_{l-1,1} \, \cgcoeff^{l,m_s}_{l-1,1} \Big].
\end{align}

We can similarly yield the relations for the perpendicular gradients:
\begin{align}
x \,\gradpYlm &= \almperp \gradpY^{m+1}_{l+1} + \blmperp \gradpY^{m+1}_{l-1} + \dlmperp \gradpY^{m-1}_{l+1} + \elmperp \gradpY^{m-1}_{l-1} \nonumber \\
& \quad \quad \quad + \hlmperp \gradY^{m+1}_{l} + \jlmperp \gradY^{m-1}_{l}, \\
y \,\gradpYlm &= i \, \Big[ -\almperp \gradpY^{m+1}_{l+1} - \blmperp \gradpY^{m+1}_{l-1} + \dlmperp \gradpY^{m-1}_{l+1} + \elmperp \gradpY^{m-1}_{l-1} \nonumber \\
& \quad \quad \quad - \hlmperp \gradY^{m+1}_{l} + \jlmperp \gradY^{m-1}_{l}, \\
z \,\gradpYlm &= \flmperp \gradpY^{m}_{l+1} + \glmperp \gradpY^{m}_{l-1} + \klmperp \gradY^{m}_{l},
\end{align}
where it can be shown that
\begin{align}
\almperp &= \alm^*, \quad \blmperp = \blm^*, \quad \dlmperp = \dlm^*, \quad \elmperp = \elm^*, \nonumber \\
\flmperp = \flm^*,& \quad \glmperp = \glm^*, \quad \hlmperp = \hlm^*, \quad \jlmperp = \jlm^*, \quad \klmperp = \klm^*,
\end{align}
where \(^*\) denotes the complex conjugate.



\subsection{Jacobi matrices}

Define \(\gradP\) as the column vector
\begin{align}
\gradP = \begin{bmatrix} \gradP_0 \\ \gradP_1 \\ \vdots \end{bmatrix}, \quad \text{where } \quad \gradP_l = \begin{bmatrix} \gradY^{-l}_l \\ \gradpY^{-l}_l \\ \vdots \\ \gradY^{l}_l \\ \gradpY^{l}_l  \end{bmatrix} \quad \forall \: l \in \No.
\end{align}

Then we can define the Jacobi operators \(J_\nabla^x, J_\nabla^y, J_\nabla^z\) by 
\begin{align}
J_\nabla^x \gradP = x \gradP, \quad J_\nabla^y \gradP = y \gradP, \quad J_\nabla^z \gradP = z \gradP,
\end{align}
where each entry \(\gradYlm\), \(\gradpYlm\) is psuedo-treated as a single element of the vector \(\gradP\).

The Jacobi matrices have the following form:
\begin{align}
J_\nabla^x &= \begin{bmatrix}
		B^x_0 & A^x_0 & & & & \\
		C^x_1 & B^x_1 & A^x_1 & & & \\
		& C^x_2 & B^x_2 & A^x_2  & & & \\
		& & C^x_3 & \ddots & \ddots & \\
		& & & \ddots & \ddots & \ddots \\
	\end{bmatrix}, \\
J_\nabla^y &= \begin{bmatrix}
		B^y_0 & A^y_0 & & & & \\
		C^y_1 & B^y_1 & A^y_1 & & & \\
		& C^y_2 & B^y_2 & A^y_2  & & & \\
		& & C^y_3 & \ddots & \ddots & \\
		& & & \ddots & \ddots & \ddots \\
	\end{bmatrix}, \\
J_\nabla^z &= \begin{bmatrix}
		B^z_0 & A^z_0 & & & & \\
		C^z_1 & B^z_1 & A^z_1 & & & \\
		& C^z_2 & B^z_2 & A^z_2  & & & \\
		& & C^z_3 & \ddots & \ddots & \\
		& & & \ddots & \ddots & \ddots \\
	\end{bmatrix},
\end{align}
where for \(l \in \No\),

\begin{align}
A^x_l &:= \begin{bmatrix}
		d_{l,-l} & 0 & 0 & 0 & a_{l,-l} & & \\
		& d^\perp_{l,-l} & 0 & 0 & 0 & a_{l,-l} & & \\
		& & \ddots & \ddots & \ddots & \ddots & \ddots & \\
		& & & d_{l,l} & 0 & 0 & 0 & a_{l,l} \\
		& & & & d^\perp_{l,l} & 0 & 0 & 0 & a^\perp_{l,-l}  \\
	    \end{bmatrix} \in \R^{2(2l+1)\times2(2l+3)}, 
\end{align}
\begin{align}
B^x_l &:= \begin{bmatrix}
		0 & 0 & 0 & h_{l,-l} & & \\
		0 & 0 & h^\perp_{l,-l} & 0 & & \\
		0 & j_{l,-l+1} & 0 & 0 & 0 & h_{l,-l+1} \\
		j^\perp_{l,-l+1} & 0 & 0 & 0 & h^\perp_{l,-l+1} & 0 \\
		& \ddots & & \ddots & & \ddots \\
		& & \ddots & & \ddots & & \ddots \\
		& & 0 & j_{l,l-1} & 0 & 0 & 0 & h_{l,l-1} \\
		& & j^\perp_{l,l-1} & 0 & 0 & 0 & h^\perp_{l,l-1} & 0 \\
		& & & & 0 & j_{l,l} & 0 & 0 \\
		& & & & j^\perp_{l,l} & 0 & 0 & 0 \\
	\end{bmatrix}  \in \R^{2(2l+1)\times2(2l+1)}, 
\end{align}
\begin{align}
C^x_l &:= \begin{bmatrix}
		b_{l,-l} & & \\
		0 & b^\perp_{l,-l} \\
		0 & 0 & \ddots & \\
		0 & 0 & \ddots & \ddots \\
		e_{l,-l+2} & 0 & \ddots & \ddots & b_{l,l-2} \\
		& e^\perp_{l,-l+2} & \ddots & \ddots & 0 & b^\perp_{l,l-2} \\
		& & \ddots & \ddots & 0 & 0 \\
		& & & \ddots & 0 & 0 \\
		& & & & e_{l,l} & 0 \\
		& & & & & e^\perp_{l,l} \\
	    \end{bmatrix} \in \R^{2(2l+1)\times2(2l-1)} \quad (l \ne 0),
\end{align}
\begin{align}
A^y_l &:= \begin{bmatrix}
		-d_{l,-l} & 0 & 0 & 0 & a_{l,-l} & & \\
		& -d^\perp_{l,-l} & 0 & 0 & 0 & a_{l,-l} & & \\
		& & \ddots & \ddots & \ddots & \ddots & \ddots & \\
		& & & -d_{l,l} & 0 & 0 & 0 & a_{l,l} \\
		& & & & -d^\perp_{l,l} & 0 & 0 & 0 & a^\perp_{l,-l}  \\
	    \end{bmatrix} \in \R^{2(2l+1)\times2(2l+3)},
\end{align}
\begin{align}
B^y_l &:= \begin{bmatrix}
		0 & 0 & 0 & h_{l,-l} & & \\
		0 & 0 & h^\perp_{l,-l} & 0 & & \\
		0 & -j_{l,-l+1} & 0 & 0 & 0 & h_{l,-l+1} \\
		-j^\perp_{l,-l+1} & 0 & 0 & 0 & h^\perp_{l,-l+1} & 0 \\
		& \ddots & & \ddots & & \ddots \\
		& & \ddots & & \ddots & & \ddots \\
		& & 0 & -j_{l,l-1} & 0 & 0 & 0 & h_{l,l-1} \\
		& & -j^\perp_{l,l-1} & 0 & 0 & 0 & h^\perp_{l,l-1} & 0 \\
		& & & & 0 & -j_{l,l} & 0 & 0 \\
		& & & & -j^\perp_{l,l} & 0 & 0 & 0 \\
		\end{bmatrix}  \in \R^{2(2l+1)\times2(2l+1)}, 
\end{align}
\begin{align}
C^y_l &:= \begin{bmatrix}
		b_{l,-l} & & \\
		0 & b^\perp_{l,-l} \\
		0 & 0 & \ddots & \\
		0 & 0 & \ddots & \ddots \\
		-e_{l,-l+2} & 0 & \ddots & \ddots & b_{l,l-2} \\
		& -e^\perp_{l,-l+2} & \ddots & \ddots & 0 & b^\perp_{l,l-2} \\
		& & \ddots & \ddots & 0 & 0 \\
		& & & \ddots & 0 & 0 \\
		& & & & -e_{l,l} & 0 \\
		& & & & & -e^\perp_{l,l} \\
	    \end{bmatrix} \in \R^{2(2l+1)\times2(2l-1)} \quad (l \ne 0),
\end{align}
\begin{align}
A^z_l &:= \begin{bmatrix}
		0 & 0 & f_{l,-l} \\
		& & & f^\perp_{l,-l} \\
		& & & & \ddots \\
		& & & & & f_{l,l} \\
		& & & & & & f^\perp_{l,l} & 0 & 0 \\
	    \end{bmatrix} \in \R^{2(2l+1)\times2(2l+3)}, \\
\end{align}
\begin{align}
B^z_l &:= \begin{bmatrix}
		0 & k_{l,-l} \\
		k^\perp_{l,-l} & 0 \\
		& & \ddots \\
		& & & 0 & k_{l,l} \\
		& & & k^\perp_{l,l} & 0 \\
	\end{bmatrix}  \in \R^{2(2l+1)\times2(2l+1)}, 
\end{align}
\begin{align}
C^z_l &:= \begin{bmatrix}
		0 \\
		0 \\
		g_{l,-l+1} \\
		& g^\perp_{l,-l+1} \\
		& & \ddots \\
		& & & g_{l,l-1} \\
		& & & & g^\perp_{l,l-1} \\
		& & & & 0 \\
		& & & & 0 \\
	    \end{bmatrix} \in \R^{2(2l+1)\times2(2l-1)} \quad (l \ne 0). \\
\end{align}



\subsection{Three-term recurrence relation for \(\gradP\)}

Combining each system in (102) we can write, for \((x,y,z)\) on the unit sphere, the tridiagonal-block system
\begin{align}
\renewcommand\arraystretch{1.3}
\mleft[
\begin{array}{c|c|c|cc}
		I_{2}  & & & \\
		\hline
		& I_{6}  & & & \\
		\hline 
		& B^x_1 - xI_6 & A^x_1 & & \\
		& B^y_1 - yI_6 & A^y_1 & & \\
		& B^z_1 - zI_6 & A^z_1 & & \\
		\hline 
		& C^x_2 & B^x_2 - xI_{10} & \quad A^x_2 \quad & \\
		& C^y_2 & B^y_2 - yI_{10} & \quad A^y_2 \quad & \\
		& C^z_2 & B^z_2 - zI_{10} & \quad A^z_2 \quad & \\
		\hline
		& & C^x_3 & \ddots & \ddots \\
		& & C^y_3 & & \\
		& & C^z_3 & & \\
		\hline
		& & & \ddots &
\end{array}
\mright]
\gradP
=
\begin{bmatrix}
	\underline{0}_3 \\ \underline{0}_3 \\ \gradP_1 \\ \underline{0}_3 \\ \\ \vdots \\ \\ \vdots \\ \\ \vdots \\ \\
\end{bmatrix},
\end{align}
where \(I_{2l+1}\) is the \((2l+1)\times(2l+1)\) identity matrix. For clarity we again treat each sub-vector \(\gradYlm, \gradpYlm\) of \(\gradP\) in the matrix-vector multiplication as a single element.

Let us define the joint matrices that comprise each block. For each \(l \in \N\):
\begin{align}
A_l &:= \begin{bmatrix}
		A^x_l \\
		A^y_l \\
		A^z_l
	    \end{bmatrix} \in \R^{6(2l+1)\times2(2l+3)}, \quad
C_l := \begin{bmatrix}
		C^x_l \\
		C^y_l \\
		C^z_l
	    \end{bmatrix} \in \R^{6(2l+1)\times2(2l-1)} \quad (l \ne 1), \\
B_l &:= \begin{bmatrix}
		B^x_l \\
		B^y_l \\
		B^z_l
	    \end{bmatrix} \in \R^{6(2l+1)\times2(2l+1)}, \quad
G_l(x,y,z) := \begin{bmatrix}
		xI_{2l+1} \\
		yI_{2l+1} \\
		zI_{2l+1}
	    \end{bmatrix} \in \R^{6(2l+1)\times2(2l+1)}.
\end{align}
Then our system simply becomes
\begin{align}
\renewcommand\arraystretch{1.3}
\mleft[
\begin{array}{cccc}
		I_2 \\
		0 & I_6 & & \\
		D^T_1 C_1 & D^1_1(B_1-G_1) & A_1 \quad & \\
		& D^T_2 C_2 & \ddots & \ddots \\
		& & \ddots &
\end{array}
\mright]
\bigP
=
\begin{bmatrix}
	\underline{0}_3 \\ \underline{0}_3 \\ \gradP_1 \\ \underline{0}_3 \\ \\ \vdots \\
\end{bmatrix}.
\end{align}

For each \(l \in \N\) let \(\Dlt\) be any matrix that is a left inverse of \(A_l\), i.e. such that \(\Dlt A_l = I_{2(2l+3)}\). Multiplying our system by the preconditioner matrix that is given by the block diagonal matrix of the \(\Dlt\)'s, we obtain the lower triangular system
\begin{align}
\renewcommand\arraystretch{1.3}
\mleft[
\begin{array}{cccc}
		I_2 \\
		0 & I_6 & & \\
		D^T_1 C_1 & D^T_1(B_1-G_1) & \quad I_{10} \quad & \\
		& D^T_2 C_2 & \ddots & \ddots \\
		& & \ddots &
\end{array}
\mright]
\bigP
=
\begin{bmatrix}
	\underline{0}_3 \\ \underline{0}_3 \\ \gradP_1 \\ \underline{0}_3 \\ \\ \vdots \\
\end{bmatrix}.
\end{align}

Expanding this we can find a three-term recurrence relation for each \(\gradP_{l+1}\) in terms of the previous two sub-vectors \(\gradPl\) and \(\gradP_{l-1}\):
\begin{align}
\begin{cases}
\gradP_{0}(x,y,z) := \underline{0}_6 \\
\\
\gradP_{1}(x,y,z) := \begin{bmatrix}
				\gradY_1^{-1}(x,y,z) \\ \gradpY_1^{-1}(x,y,z) \\ \gradY_1^{0}(x,y,z) \\ \gradpY_1^{0}(x,y,z) \\ \gradY_1^{1}(x,y,z) \\ \gradpY_1^{1}(x,y,z) \\
			\end{bmatrix} \\
\\
\gradP_{l+1}(x,y,z) = -\Dlt [B_l-G_l(x,y,z)] \gradPl(x,y,z) - \Dlt C_l  \,\gradP_{l-1}(x,y,z), \quad l \in \N.
\end{cases}
\end{align}

We note that we can choose the matrices \(\Dlt\) in the following way. For \(l \in \N\), we set
\begin{align}
\Dlt = \begin{bmatrix}
		\hat{A}^{x,y}_l & 0_{2(2l+3)\times2(2l+1)}
	  \end{bmatrix} \in \R^{2(2l+3)\times6(2l+1)}
\end{align}
where \(0_{2(2l+3)\times2(2l+1)}\) the zero matrix in \(\R^{2(2l+3)\times2(2l+1)}\), and \(\hat{A}^{x,y}_l \in \R^{2(2l+3)\times4(2l+1)}\) is the left inverse of the matrix \(\begin{bmatrix} A^x_l \\ A^y_l \end{bmatrix}\), given by
\begin{align}
\hat{A}^{x,y}_l = \begin{bmatrix}
		\frac{1}{2d_{l,-l}} & 0 & \hdots & 0 & -\frac{i}{2d_{l,-l}} & 0 & \hdots &  \\
		0 & \frac{1}{2d^\perp_{l,-l}} & 0 & \hdots & 0 & -\frac{i}{2d^\perp_{l,-l}} & 0 & \hdots &  \\
		& & \ddots & & & & \ddots & & \\
		\vdots & & & \ddots & & & & \ddots & \\
		& & & & \frac{1}{2d_{l,l}} & 0 & \hdots & 0 & -\frac{i}{2d_{l,l}} & \\
		& & & & & \frac{1}{2d^\perp_{l,l}} & 0 & \hdots & 0 & -\frac{i}{2d^\perp_{l,l}} \\
		0 & \hdots & \frac{1}{2a_{l,l-1}} & 0 & \hdots & 0 & \frac{i}{2a_{l,l-1}} & 0 & 0 & 0 \\
		0 & \hdots & 0 & \frac{1}{2a^\perp_{l,l-1}} & 0 & \hdots & 0 & \frac{i}{2a^\perp_{l,l-1}} & 0 & 0\\
		0 & \hdots & \hdots & 0 & \frac{1}{2a_{l,l}} & 0 & \hdots & 0 & \frac{i}{2a_{l,l}} & 0 \\
		0 & \hdots & \hdots & & 0 & \frac{1}{2a^\perp_{l,l}} & 0 & \hdots & 0 & \frac{i}{2a^\perp_{l,l}} \\
	  \end{bmatrix}.
\end{align}



\subsection{Deriving matrices and calculations for certain operations}

Define \(\unitvec\) as the unit outward normal vector at the point on the sphere \((x,y,z)\), so that
\begin{align}
\unitvec = \begin{bmatrix} x \\ y \\ z \end{bmatrix}.
\end{align}

Let \(S := \{ \bold{x} := (x,y,z) \quad | \quad ||\bold{x}|| = 1\}\) be the unit sphere in \(\R^3\) and let \(T_x^S\) denote the tangent space at the point \(\bold{x} \in S\). Further, let \(\bold{u}(x,y,z)\), \(\bold{v}(x,y,z)\) be two vector valued functions for \((x,y,z)\) on the unit sphere with values in the tangent space (representing the tangential velocity of a flow for example) and let \(h(x,y,z)\) be a scalar function on the sphere.

Then, 
\begin{align}
\bold{u} &= \sum_{l=0}^{\infty} \sum_{m=-l}^{l} [ u_{l,m} \gradYlm + u^{\perp}_{l,m} \gradpYlm ], \\
\bold{v} &= \sum_{l=0}^{\infty} \sum_{m=-l}^{l} [ v_{l,m} \gradYlm + v^{\perp}_{l,m} \gradpYlm ], \\
h &= \sum_{l=0}^{\infty} \sum_{m=-l}^{l} h_{l,m} \Ylm,
\end{align}
for some real coefficients \(u_{l,m}, u^{\perp}_{l,m}, v_{l,m}, v^{\perp}_{l,m}, h_{l,m}\). Define the vectors of these coefficients as 
\begin{align}
\bold{u}^c &= \begin{bmatrix}
			\bold{u}_0^c \\
			\bold{u}_1^c \\
			\vdots \\
		    \end{bmatrix},
\quad
\bold{v}^c = \begin{bmatrix}
			\bold{v}_0^c \\
			\bold{v}_1^c \\
			\vdots \\
		    \end{bmatrix},
\quad
\bold{h}^c = \begin{bmatrix}
			\bold{h}_0^c \\
			\bold{h}_1^c \\
			\vdots \\
		    \end{bmatrix},		  
\end{align}
where
\begin{align}
\bold{u}^c_l &= \begin{bmatrix}
				u_{l,-l} \\
				u^{\perp}_{l,-l} \\
				\vdots \\
				u_{l,l} \\
				u^{\perp}_{l,l}
		        \end{bmatrix},
\quad
\bold{v}^c_l = \begin{bmatrix}
				v_{l,-l} \\
				v^{\perp}_{l,-l} \\
				\vdots \\
				v_{l,l} \\
				v^{\perp}_{l,l}
		        \end{bmatrix},
\quad
\bold{h}^c_l = \begin{bmatrix}
				h_{l,-l} \\
				\vdots \\
				h_{l,l} \\
		        \end{bmatrix},
\quad \forall l \in \No	  
\end{align}.

Then, for large enough \(N \in \N\) we have that
\begin{align}
\bold{u} &\approx \sum_{l=0}^{N} \sum_{m=-l}^{l} [ u_{l,m} \gradYlm + u^{\perp}_{l,m} \gradpYlm ], \\
\bold{v} &\approx \sum_{l=0}^{N} \sum_{m=-l}^{l} [ v_{l,m} \gradYlm + v^{\perp}_{l,m} \gradpYlm ], \\
h &\approx \sum_{l=0}^{N} \sum_{m=-l}^{l} h_{l,m} \Ylm,
\end{align}
and so we define the truncated coefficient vectors for some \(N \in \N\) as
\begin{align}
\bold{u}^c &= \begin{bmatrix}
			\bold{u}_0^c \\
			\bold{u}_1^c \\
			\vdots \\
			\bold{u}_N^c
		    \end{bmatrix},
\quad
\bold{v}^c = \begin{bmatrix}
			\bold{v}_0^c \\
			\bold{v}_1^c \\
			\vdots \\
			\bold{v}_N^c
		    \end{bmatrix},
\quad
\bold{h}^c = \begin{bmatrix}
			\bold{h}_0^c \\
			\bold{h}_1^c \\
			\vdots \\
			\bold{h}_N^c
		    \end{bmatrix}.		  
\end{align}

We note that
\begin{align}
\gradpYlm = \unitvec \times \gradYlm, \quad \forall \; l \in \No, \; m \in \Z \; \text{s.t.} \; |m| \le l.
\end{align}


\subsubsection{Operator for "\(\unitvec \times \quad\)"}
\begin{align}
\unitvec \times \bold{u} &= \unitvec \times \sum_{l=0}^{\infty} \sum_{m=-l}^{l} [ u_{l,m} \gradYlm + u^{\perp}_{l,m} \gradpYlm ] \\
&= \sum_{l=0}^{\infty} \sum_{m=-l}^{l} [ u_{l,m} \unitvec \times \gradYlm + u^{\perp}_{l,m} \unitvec \times \gradpYlm ] \\
&= \sum_{l=0}^{\infty} \sum_{m=-l}^{l} [ u_{l,m} \unitvec \times \gradYlm + u^{\perp}_{l,m} \unitvec \times ( \unitvec \times \gradYlm ) ] \\
&= \sum_{l=0}^{\infty} \sum_{m=-l}^{l} [ u_{l,m} \gradpYlm + u^{\perp}_{l,m} (( \unitvec \cdot \gradpYlm) \unitvec - ( \unitvec \cdot \unitvec ) \gradYlm ) ] \\
&= \sum_{l=0}^{\infty} \sum_{m=-l}^{l} [ u_{l,m} \gradpYlm - u^{\perp}_{l,m} \gradYlm ) ]. \\
\end{align}

Thus the operator matrix for the cross product from the left by the normal unit vector at the point \((x,y,z)\) is given by
\begin{align}
K =  \begin{bmatrix}
		0 & -1 & & & & \\
		1 & 0 & & & & \\
		& & 0 & -1 & & \\
		& & 1 & 0 & & \\
		& & & & \ddots \\
		& & & & & \ddots 
	  \end{bmatrix} \quad \in \quad \R^{2(N+1)^2 \times 2(N+1)^2}.
\end{align}
so that \(K\bold{u}^c\) gives the vector of coefficients for for expansion of \(\unitvec \times \bold{u}\) in the vector spherical harmonic basis (up to degree N).


\subsubsection{Operator for Div}
\begin{align}
\nabla \cdot \bold{u} &= \nabla \cdot \Big( \sum_l \sum_{m=-l}^{l} [ u_{l,m} \gradYlm + u^{\perp}_{l,m} \gradpYlm ] \Big) \\
&= \sum_l \sum_{m=-l}^{l} [ u_{l,m} \: \Delta \Ylm + u^{\perp}_{l,m} \nabla \cdot (\unitvec \times \gradYlm) ] \\
&= \sum_l \sum_{m=-l}^{l} u_{l,m} \: \Delta \Ylm \\
&= \sum_l \sum_{m=-l}^{l} - u_{l,m} \: l(l+1) \Ylm, \\
\end{align}
using the fact that 
\begin{align}
\Delta \Ylm(x,y,z) = -l(l+1)\Ylm(x,y,z), \quad \forall \; l \in \No, \; m \in \Z \; \text{s.t.} \; |m| \le l.
\end{align}

Thus, the operator matrix for the divergence of a vector in the tangent space is
\begin{align}
D &=  \begin{bmatrix}
		-l(l+1)|_{l=0} & 0 & & & & & & & \\
		& & -l(l+1)|_{l=1} & 0 & & & & & \\
		& & & & -l(l+1)|_{l=1} & 0 & & & \\
		& & & & & & -l(l+1)|_{l=1} & 0 & \\
		& & & & & & & \ddots 
	  \end{bmatrix} \\
	  &=  \begin{bmatrix}
		0 & 0 & & & & & & & \\
		& & -2 & 0 & & & & & \\
		& & & & -2 & 0 & & & \\
		& & & & & & -2 & 0 & \\
		& & & & & & & \ddots 
	  \end{bmatrix} \quad \in \quad \R^{(N+1)^2 \times 2(N+1)^2},
\end{align}
so that \(D\bold{u}^c\) gives the vector of coefficients of the expansion for \(\nabla \cdot u\) in the spherical harmonic basis.


\subsubsection{Operator for Grad}
\begin{align}
\nabla h &= \nabla \Big( \sum_l \sum_{m=-l}^{l} h_{l,m} \Ylm \Big) \\
&= \sum_l \sum_{m=-l}^{l} h_{l,m} \gradYlm
\end{align}

Thus, the operator matrix for the gradient of a scalar function on the sphere is
\begin{align}
G &=  \begin{bmatrix}
		\tilde{G_0} & & & \\
		& \tilde{G_1} & & \\
		& & \tilde{G_2} & \\
		& & & \ddots
	  \end{bmatrix}  \quad \in \quad \R^{2(N+1)^2 \times (N+1)^2},
\end{align}
where
\begin{align}
\tilde{G_l} &=  \begin{bmatrix}
		1 & & & \\
		0 & & & \\
		& 1 & & \\
		& 0 & & \\
		& & 1 & \\
		& & 0 & \\
		& & & \ddots \\
	  \end{bmatrix} \quad \in \quad \R^{2(2l+1) \times (2l+1)},
\end{align}
so that \(G\bold{h}^c\) gives the vector of coefficients of the expansion for \(\nabla h\) in the tangent space (vector spherical harmonic) basis.


\subsubsection{Dot products}

We consider the general case that \(\bold{u}\) is expanded up to order \(N_1 \in \N\) and \(\bold{v}\) is expanded up to order \(N_2 \in \N\), i.e. 
\begin{align}
\bold{u}^c &= \begin{bmatrix}
			\bold{u}_0^c \\
			\bold{u}_1^c \\
			\vdots \\
			\bold{u}_{N_1}^c
		    \end{bmatrix},
\quad
\bold{v}^c = \begin{bmatrix}
			\bold{v}_0^c \\
			\bold{v}_1^c \\
			\vdots \\
			\bold{v}_{N_2}^c
		    \end{bmatrix}.
\end{align}

Let \((x,y,z)\) lie on the unit sphere and let \(l = 1\), \(m \in {-1, 0, 1}\) and define \(\tilde{N} := N_1 + N_2\). Then
\begin{align}
\gradYlm(x,y,z) \cdot \bold{v}(x,y,z) = (\bold{b}^c)^T  \: \bigP(x,y,z)
\end{align}
where
\begin{align}
\bigP &:= \begin{bmatrix}
		\bigP_0 \\
		\bigP_1 \\
		\bigP_2 \\
		\vdots \\
		\bigP_{N_2}
	  \end{bmatrix}, \quad 
\Pl := \begin{bmatrix}
		Y_l^{-l} \\
		\vdots \\	
		Y_l^l
	  \end{bmatrix},
\end{align}
for some coefficient vector \(\bold{b}^c \in \R^{(N_2+1)^2}\). In other words, the dot product of one of the \(l=1\) tangent space's orthogonal polynomials with a vector valued function on the unit sphere can naturally be written as an expansion in the scalar spherical harmonic OP basis. Similarly,
\begin{align}
\gradpYlm(x,y,z) \cdot \bold{v}(x,y,z) = (\bold{b^\perp}^c)^T  \: \bigP(x,y,z)
\end{align}
for some coefficient vector \(\bold{b^\perp}^c \in \R^{(N_2+2)^2}\).

Then, 
\begin{align}
\bold{b}^c &= J_{l,m} \: \bold{v}^c, \quad 
\bold{b^\perp}^c = J^\perp_{l,m} \: \bold{v}^c
\end{align}
where \(J_{l,m}\), \(J^\perp_{l,m}\) are (operator) matrices for the dot product with \(\gradYlm(x,y,z)\) and \(\gradpYlm(x,y,z)\) respectively.
Each \(J_{l,m}\), \(J^\perp_{l,m}\) will have the \((\tilde{N} \times N_2)\) block structure:
\begin{align}
	\begin{array}{c|c|c|c}
		1 \times 6 & 1 \times 10 & \\
		\hline
		3 \times 6 & 3 \times 10 & 3 \times 14 & \\
		\hline
		5 \times 6 & 5 \times 10 & 5 \times 14 & \ddots \\
		\hline
		&  7 \times 10 & 7 \times 14 & \ddots \\
		\hline
		& & \ddots & \ddots \\
	  \end{array}
\end{align}
i.e. a BandedBlockBandedMatrix\((J, (rows, cols), (2,1), (?,?))\) [note: I am not sure what the sub-block bands should be] where 
\begin{align}
rows &= 1:2:2\tilde{N}+1 \\
cols &= 6:4:2(2N_2+1).
\end{align}

Define
\begin{align}
T^\bigP_0 &:= 0, \quad
T^\bigP_1 := \begin{bmatrix}
			J_{1,-1} \\
			J^\perp_{1,-1} \\
			J_{1,0} \\
			J^\perp_{1,0} \\
			J_{1,1} \\
			J^\perp_{1,1}
	  	\end{bmatrix}.
\end{align}
Then by linearity of the dot product, we can use the recurrence relation (122) for \(\gradPl\) to gain a recurrence for \(T^\bigP_l\):
\begin{align}
T^\bigP_{l+1} = -\Dlt [B_l-G_l(J^x, J^y, J^z)] \: T^\bigP_l - \Dlt C_l  \, T^\bigP_{l-1}, \quad l \in \{1,\dots,N_1-1\},
\end{align}
where \(J^x, J^y, J^z\) are the \(\tilde{N} \times \tilde{N}\) Jacobi operator matrices for multiplication of the scalar spherical harmonic basis by \(x, y, z\) respectively. We then have that for each \(l \in {1,\dots,N_1}\),
\begin{align}
T^\bigP_l := \begin{bmatrix}
			J_{l,-l} \\
			J^\perp_{l,-l} \\
			\vdots \\
			J_{l,l} \\
			J^\perp_{l,l}
	  	\end{bmatrix}.
\end{align}

Then, 
\begin{align}
\bold{u}(x,y,z) \cdot \bold{v}(x,y,z) = (\bold{b}^c)^T  \: \bigP(x,y,z)
\end{align}
where \(\bold{b}^c \in \R^{(\tilde{N}+1)^2}\) and is given by
\begin{align}
\bold{b}^c = \Big(\sum_{l=0}^{N_1} \sum_{m=-l}^{l} [ u_{l,m} J_{l,m} + u^{\perp}_{l,m} J^\perp_{l,m} ] \Big) \: \bold{v}^c.
\end{align}


% \bibliography{spherical-harmonics-bib}


\end{document}











  